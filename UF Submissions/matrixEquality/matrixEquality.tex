\documentclass{ximera}
%../preamble.tex
\title{Matrix Equality}

\begin{document}

\begin{abstract}
    An attempt to develop a format for student answers, to allow students to enter a matrix and validate the answer as correct if it equals the instructor's matrix.
\end{abstract}
\maketitle

This validator is designed to ensure that a matrix submitted by a student is the same as the matrix submitted by the instructor. This also establishes the format for how matricies should be entered into the answer box and answer command.

NOTE:: This validator is still under construction. Currently this validator (seems to) work on algebraic entries of any kinda that Xronos normally supports.

Specifically, a matrix should be entered into the answer box in the following format:

$[ [r1c1,r1c2,r1c3],[r2c1,r2c2,r2c3],[r3c1,r3c2,r3c3] ]$.

So, if a student wishes to enter the matrix:

$\left[\begin{matrix}
1 & 2 & 3 \\
4 & 5 & 6 \\
7 & 8 & 9
\end{matrix}\right]$

They would enter into the answer prompt: $[ [1,2,3], [4,5,6], [7,8,9] ]$.\\



\begin{javascript}

// We want to define a validator that allows us to check a matrix.
function equalMatrix(f,g) {
    /* We expect that f is a matrix provided by the student in the format:
            [ [r1c1,r1c2,r1c3],[r2c1,r2c2,r2c3],[r3c1,r3c2,r3c3] ]
       We expect that g is a matrix provided by the instructor in the format:
            [ [r1c1,r1c2,r1c3],[r2c1,r2c2,r2c3],[r3c1,r3c2,r3c3] ]
    */
    
    
    debugText('Here is the student ans.');
    
    debugArray(f);
    
    debugText('Here is the instructor ans.');
    
    debugArray(g);
    
    //  Now we want to make a dummy form of the given variables so we can manipulate them without the original getting recked.
   
    var fTree = JSON.parse(JSON.stringify(f.tree));
    var gTree = JSON.parse(JSON.stringify(g.tree));
    
    
    // Start out by making sure that the format was entered correctly
    //  in particular, that each sub-list is the same length.
    //instructorAnsbkup = JSON.parse(JSON.stringify(g.tree));
    
    // Santize the instructor's entry.
    
    for (var i = 1; i < g.tree.length; ++i) {
        if (g.tree[1].length==g.tree[i].length) {
        } else {
            console.log('Looks like the format for the instructors answer is off, [at least] one of the rows has the wrong number of columns!');
            return false
        }
    }

    // Santize the student entry.
    for (var i = 1; i < f.tree.length; ++i) {
        if (f.tree[1].length==f.tree[i].length) {
        } else {
            console.log('Looks like the format for the students answer is off, [at least] one of the rows has the wrong number of columns!');
            return false
        }
    }
    
    // Now that we have santized the column/rows so that we know we have an actual rectangle, we'll check that the dimensions match.
    
    if (f.tree.length!==g.tree.length) {
        // If the dimensions aren't the same, then the answer definitely doesn't match the instructors.
        debugText('Looks like the matrix submitted by the student does not have the same dimension as the matrix we expect.');
        return false;
    }
    
    for (var i = 1; i < f.tree.length; ++i) {
        if (f.tree[i].length==g.tree[i].length) {
        } else {
            debugText('Looks like the matrix submitted by the student does not have the same dimension as the matrix we expect.');
            return false
        }
    }
    
    // Finally, we check the actual entries. 
    // We assume that the matrix is a two dimensional array, i.e. no matrix entries inside our matrix. Tensors can bite me.
    
    for (var i = 1; i < f.tree.length; ++i) {
        for (var j = 1; j < f.tree[i].length; ++j) {
            debugText('Checking equality of entries in: row ' + i +', column ' + j);
            
            if ( compSubTree(f,fTree[i][j],gTree[i][j]) ) {
                debugText('They appear equal in student and instructors matricies.');
            } else {
                console.log('Looks like [at least one] of the entries are wrong!');
                debugText('The students answers [raw ans then copied obj ans]')
                debugArray(f.tree[i][j]);
                debugArray(fTree);
                debugText('The students answers [raw ans then copied obj ans]')
                debugArray(g.tree[i][j]);
                debugArray(gTree);
                return false
            }
        }
    }
    
return true
}


\end{javascript}

\begin{problem}

The following answer box should accept the matrix:

$\left[\begin{matrix}
a^2 & 2 & 3 \\
4 & x^2 & 6 \\
7 & 8 & y-x+e^x
\end{matrix}\right]$

$\answer[validator=equalMatrix]{[ [a^2,2,3], [4,x^2,6], [7,8,y-x+e^x] ]}$
    
\end{problem}




\end{document}
