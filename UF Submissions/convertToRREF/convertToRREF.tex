\documentclass{ximera}
%\newcommand{\dfn}{\textbf}
\renewcommand{\vec}[1]{{\overset{\boldsymbol{\rightharpoonup}}{\mathbf{#1}}}\hspace{0in}}
%% Simple horiz vectors
\renewcommand{\vector}[1]{\left\langle #1\right\rangle}
\newcommand{\arrowvec}[1]{{\overset{\rightharpoonup}{#1}}}
\newcommand{\R}{\mathbb{R}}
\newcommand{\transpose}{\intercal}






\usepackage{mdframed} % For framing content
%\usepackage{ifthen}   % For conditional statements

% Define the 'concept' environment with an optional header
\newenvironment{concept}[1][]{%
  \begin{mdframed}[linecolor=black, linewidth=2pt, innertopmargin=5pt, innerbottommargin=5pt, skipabove=12pt, skipbelow=12pt]%
    \noindent\large\textbf{#1}\normalsize%
}{%
  \end{mdframed}%
}











\colorlet{textColor}{black}
\colorlet{background}{white}
\colorlet{penColor}{blue!50!black} % Color of a curve in a plot
\colorlet{penColor2}{red!50!black}% Color of a curve in a plot
\colorlet{penColor3}{red!50!blue} % Color of a curve in a plot
\colorlet{penColor4}{green!50!black} % Color of a curve in a plot
\colorlet{penColor5}{orange!80!black} % Color of a curve in a plot
\colorlet{penColor6}{yellow!70!black} % Color of a curve in a plot
\colorlet{fill1}{penColor!20} % Color of fill in a plot
\colorlet{fill2}{penColor2!20} % Color of fill in a plot
\colorlet{fillp}{fill1} % Color of positive area
\colorlet{filln}{penColor2!20} % Color of negative area
\colorlet{fill3}{penColor3!20} % Fill
\colorlet{fill4}{penColor4!20} % Fill
\colorlet{fill5}{penColor5!20} % Fill
\colorlet{gridColor}{gray!50} % Color of grid in a plot



\usepackage[utf8]{inputenc}

% 201908/202301: PAS OP: babel en doclicense lijken problemen te veroorzaken in .jax bestand
% (wegens syntax error met toegevoegde \newcommands ...)
\pdfOnly{
    \usepackage[type={CC},modifier={by-nc-sa},version={4.0}]{doclicense}
    \usepackage[dutch]{babel}
}

%
% define softer blue/red/green, use KU Leuven base colors for blue (and dark orange for red ?)
%
% todo: rather redefine blue/red/green ...?
%\definecolor{xmblue}{rgb}{0.01, 0.31, 0.59}
%\definecolor{xmred}{rgb}{0.89, 0.02, 0.17}
\definecolor{xmdarkblue}{rgb}{0.122, 0.671, 0.835}   % KU Leuven Blauw
\definecolor{xmblue}{rgb}{0.114, 0.553, 0.69}        % KU Leuven Blauw
\definecolor{xmgreen}{rgb}{0.13, 0.55, 0.13}         % No KULeuven variant for green found ...

\definecolor{xmaccent}{rgb}{0.867, 0.541, 0.18}      % KU Leuven Accent (orange ...)
\definecolor{kuaccent}{rgb}{0.867, 0.541, 0.18}      % KU Leuven Accent (orange ...)

\colorlet{xmred}{xmaccent!50!black}                  % Darker version of KU Leuven Accent

\providecommand{\blue}[1]{{\color{blue}#1}}    
\providecommand{\red}[1]{{\color{red}#1}}


\providecommand{\isEn}{true}   % set to English

%
\ifdefined\isEn
 \newcommand{\nlen}[2]{#2}
 \newcommand{\nlentext}[2]{\text{#2}}
 \newcommand{\nlentextbf}[2]{\textbf{#2}}
\else
 \newcommand{\nlen}[2]{#1}
 \newcommand{\nlentext}[2]{\text{#1}}
 \newcommand{\nlentextbf}[2]{\textbf{#1}}
\fi

\newcommand\xmsection\subsection
\newcommand\xmsubsection\subsubsection

% Aanpassen printversie
%  (hier gedefinieerd, zodat ze in xourse kunnen worden gezet/overschreven)

\providebool{printbasicversion}
\providebool{printextendedversion}   % not properly implemented
\providebool{printfullversion}       % presumably print everything (debug/auteur)


\providebool{parttoc}
\providebool{printpartfrontpage}
\providebool{printactivitytitle}
\providebool{printactivityqrcode}
\providebool{printactivityurl}
\providebool{printcontinuouspagenumbers}
\providebool{numberactivitiesbysubpart}
\providebool{addtitlenumber}
\providebool{addsectiontitlenumber}
\addtitlenumbertrue
\addsectiontitlenumbertrue


% Commando om de printstyle toe te voegen in ximera's. Zorgt ervoor dat er geen problemen zijn als je de xourses compileert
% hack om onhandige relative paden in TeX te omzeilen
% should work both in xourse and ximera (pre-112022 only in ximera; thus obsoletes adhoc setup in xourses)
% loads global.sty if present (cfr global.css for online settings!)
% use global.sty to overwrite settings in printstyle.sty ...
\newcommand{\addPrintStyle}[1]{
\iftikzexport\else   % only in PDF
  \makeatletter
  \ifx\@onlypreamble\@notprerr\else   % ONLY if in tex-preamble   (and e.g. not when included from xourse)
    \typeout{Loading printstyle}   % logging
    \usepackage{#1/printstyle} % mag enkel geinclude worden als je die apart compileert
    \IfFileExists{#1/global.sty}{
        \typeout{Loading printstyle-folder #1/global.sty}   % logging
        \usepackage{#1/global}
        }{
        \typeout{Info: No extra #1/global.sty}   % logging
    }   % load global.sty if present
    \IfFileExists{global.sty}{
        \typeout{Loading local-folder global.sty (or TEXINPUTPATH..)}   % logging
        \usepackage{global}
    }{
        \typeout{Info: No folder/global.sty}   % logging
    }   % load global.sty if present
    \IfFileExists{\currfilebase.sty}
    {
        \typeout{Loading \currfilebase.sty}
        \input{\currfilebase.sty}
    }{
        \typeout{Info: No local \currfilebase.sty}
    }
    \fi
  \makeatother
\fi
}


\title{Convert to RREF}
\def\HCode{}

\begin{document}

\begin{abstract}
    This ``validator'' is actually a set of subfunctions to be used in validators. This reduces a numeric matrix into reduced row echelon form.
\end{abstract}
\maketitle

This ``validator'' is actually a collection of subfunctions that compute the reduced row echelon form for a numeric matrix. 


Specifically, a matrix should be entered into the answer box in the following format:

$[ [r1c1,r1c2,r1c3],[r2c1,r2c2,r2c3],[r3c1,r3c2,r3c3] ]$.

So, if a student wishes to enter the matrix:

$\left[\begin{matrix}
1 & 2 & 3 \\
4 & 5 & 6 \\
7 & 8 & 9
\end{matrix}\right]$

They would enter into the answer prompt: $[ [1,2,3], [4,5,6], [7,8,9] ]$.\\



\begin{javascript}
var debugInfo=false;
function debugText(text) {
    if (debugInfo==1) {
    console.log('DEBUG INFO::' + text)
    }
}

function debugArray(array) {
    if (debugInfo==1) {
    console.log(array)
    }
}

// We want to define a validator that allows us to check a matrix.
function equalMatrix(f,g) {
    // We expect that f is a matrix provided by the student in the format:
    //      [ [r1c1,r1c2,r1c3],[r2c1,r2c2,r2c3],[r3c1,r3c2,r3c3] ]
    // We expect that g is a matrix provided by the instructor in the format:
    //      [ [r1c1,r1c2,r1c3],[r2c1,r2c2,r2c3],[r3c1,r3c2,r3c3] ]
    //
    
    //   Debug Code - commented out now that it is working.
    debugText('Here is the student ans.');
    debugArray(f);
    
    debugText('Here is the instructor ans.');
    debugArray(g);
    
    // Start out by making sure that the format was entered correctly
    //  in particular, that each sub-list is the same length.
    //instructorAnsbkup = JSON.parse(JSON.stringify(g.tree));
    
    // Santize the instructor's entry.
    for (var i = 1; i < g.tree.length; ++i) {
        if (g.tree[1].length==g.tree[i].length) {
        } else {
            debugText('Looks like the format for the instructors answer is off, [at least] one of the rows has the wrong number of columns!');
            return false
        }
    }

    // Santize the student entry.
    for (var i = 1; i < f.tree.length; ++i) {
        if (f.tree[1].length==f.tree[i].length) {
        } else {
            debugText('Looks like the format for the students answer is off, [at least] one of the rows has the wrong number of columns!');
            return false
        }
    }
    
    // Now that we have santized the column/rows so that we know we have an actual rectangle, we'll check that the dimensions match.
    
    for (var i = 1; i < f.tree.length; ++i) {
        if (f.tree[i].length==g.tree[i].length) {
        } else {
            debugText('Looks like the matrix submitted by the student does not have the same dimension as the matrix we expect.');
            return false
        }
    }
    
    // Finally, we check the actual entries. 
    // We assume that the matrix is a two dimensional array, i.e. no matrix entries inside our matrix. Tensors can bite me.
    
    for (var i = 1; i < f.tree.length; ++i) {
        for (var j = 1; j <= f.tree[i].length; ++j) {
            if (Math.abs(f.tree[i][j]-g.tree[i][j]) < 10^(-12)) {
            } else {
                debugText('Looks like [at least one] of the entries are wrong!');
                debugArray(f.tree[i][j]);
                debugArray(g.tree[i][j]);
                return false
            }
        }
    }
return true
}


// A subfunction to build a matrix out of a given input.

function matrixForm(preForm) {
    /*
        Expects an input to be a tree. So it should have the form:
            [''array'', [''array'', 1,2,3], [''array'', 4,5,6], [''array'', 7,8,9] ]  
        Outputs a matrix form; [ [r1c1,r1c2,r1c3],[r2c1,r2c2,r2c3],[r3c1,r3c2,r3c3] ]
        Update: Apparently negatives show up as sub arrays, so we need to deal with that.
    */
    var len = preForm.tree.length;// Get the length of the initial array.
    var matrixForm=[];// Dummy matrix.
    
    // Ok, now to walk the array and build the thing so we have a sane format.
    for (var i = 1; i < len; ++i) {
        matrixForm[i-1]=[]; // initialize the subarray.
        for (var j = 1; j < preForm.tree[i].length; ++j) {
            if (Array.isArray(preForm.tree[i][j])) {
                if (preForm.tree[i][j][0]=='-'){
                    // If this is the case, the 'array' is just a negative number.
                    matrixForm[i-1][j-1] = -JSON.parse(JSON.stringify(preForm.tree[i][j][1]));
                } else {
                    // Otherwise we have an array and no idea why we have an array so ... lets bail.
                    return false;
                }
            } else {
            matrixForm[i-1][j-1]= JSON.parse(JSON.stringify(preForm.tree[i][j]));
            }
        }
    }
    debugText('Here is the matrix form:');
    debugArray(matrixForm);
    return matrixForm;
}



// A subfunction to find the row that contains the next non-zero entry in a given column, 
//  starting from a given row.

function findPivot(matrix,startRow,targetCol,avoid) {
    /*
        We expect a matrix in the form: [ [r1c1,r1c2,r1c3],[r2c1,r2c2,r2c3],[r3c1,r3c2,r3c3] ].
            A starting row number, and a column number (starting rows/cols numbers at 0, not 1).
        We want to output a number >= the given row number, which has a non-zero entry.
    */
    for (var i = startRow; i < matrix.length; ++i) {
        if ( (matrix[i][targetCol] !== 0) && (!(avoid.includes(i))) ) {
            return i;
        }
    }
    return -1;
}

// Function to make all row entries in the same col zero once you find a pivot.

function anhilCol(matrix,targetRow,targetCol) {
    /*
        We expect a matrix using the form: [ [r1c1,r1c2,r1c3],[r2c1,r2c2,r2c3],[r3c1,r3c2,r3c3] ], 
            a targetRow and a targetCol (numbers starting from 0).
        We also expect that matrix[targetRow][targetCol] is non-zero (but not necessarily one).
        We aim to return a matrix where matrix[targetRow][col]=0 for col != targetCol and 1 for targetCol.
    */
    
    // Make a dummy version of the matrix to manipulate.
    let originalMatrix = JSON.parse(JSON.stringify(matrix));
    
    // Now we make the pivot point a 1 by dividing the entire row by the value of the pivot point,
    //  i.e. newMatrix[targetRow][targetCol]
    //      BUT! We need to divide by a value, not the newMatrix entry to make sure we don't have clashing issues.
    let originalPivVal = JSON.parse(JSON.stringify(originalMatrix[targetRow][targetCol]));
    for (var i = 0; i < originalMatrix[targetRow].length; ++i) {
        matrix[targetRow][i] = originalMatrix[targetRow][i]/originalPivVal;
    }
    let newTargetRow=matrix[targetRow].slice(0);
    
    debugText('Divided out the leading value to make sure it is now 1. Now the matrix is:');
    let checkInMatrix = JSON.parse(JSON.stringify(matrix));
    debugArray(checkInMatrix);
    
    // Now we need to zero out the rest of the column entries using row operations.
    for (var i = 0; i < matrix.length; ++i) {
        if ((i !== targetRow) && (matrix[i][targetCol] !== 0)) {
            // Now we know it isn't zero or the pivot row, so we can actually kill it off.
            let leadingVal = JSON.parse(JSON.stringify(matrix[i][targetCol]));
            debugText('Trying to tackle row ' + i + ' column ' + targetCol + ' which I think has relevant col value ' + leadingVal);
            for (var j = 0; j < matrix[i].length; ++j) {
                // Now to do some sketchy stuff to fix machine percision issues.
                //  Note that this only works with numbers, so we will have to revisit this
                //  If we want to use this with algebraic expressions or something.
                matrix[i][j] = Math.round( (originalMatrix[i][j] - leadingVal*newTargetRow[j]) * (10**10))/(10**10);
                debugText('we just replaced a number with' + matrix[i][j]);
            }
        }
    }
    debugText('Ok, now we have zeroed out everything. Maybe.');
    
    debugArray(matrix);
    return matrix
}


function convertToRREF(f) {
    /*
        This function expects a numeric matrix in the normal math expressions tree format,
        and outputs the matrix in reduced row echelon form.
    */
    var debugInfo=1;
    debugText('We are starting with f.tree as...');
    debugArray(f.tree);
    var fMatrix = matrixForm(f);
    debugText('Here is the matrix form for f.');
    debugArray(fMatrix);
    debugText('And now we will try to annhilate...')
    var newMatrix = fMatrix.slice(0);
    var startingCol = 0;
    var usedRows = [];
    //  Go through each column to try and find a pivot point.
    for (var i = 0; i < fMatrix[0].length; ++i) {
        var startingRow = findPivot(newMatrix,0,i,usedRows);
        usedRows.push(startingRow);
        debugText('So, I think the new target row is ' + startingRow);
        debugArray(usedRows);
        if (startingRow >= 0) {
            // If we get a non-negative starting row, then there is a non-zero row value in that column.
            //  So we can go ahead with the row-reduction.
            debugText('starting from ' + startingRow + ',' + i);
            newMatrix = anhilCol(newMatrix,startingRow,i);
        }
    }
    debugText('And after doing it we get...');
    debugArray(newMatrix); 
    document.getElementById('rref').innerHTML = JSON.stringify(newMatrix);
    
    return newMatrix;   
}

\end{javascript}

\begin{problem}

Put in a matrix using the format: $[ [r1c1,r1c2,r1c3],[r2c1,r2c2,r2c3],[r3c1,r3c2,r3c3] ]$, and it should be converted into reduced row echelon form below. Hopefully.

$\answer[validator=convertToRREF]{[ [1,2,3], [4,5,6], [7,8,9] ]}$


\HCode{<p id='rref'>}
\HCode{</p>}


\end{problem}




\end{document}
