\documentclass{ximera}
\title{Preface and Instructions}

\begin{document}
\begin{abstract}
    An intro to the experimental repo.
\end{abstract}
\maketitle

This is a sort of defacto readme for the repo - primarily to introduce the format and intentions. Someone that actually knows how to write a nicely formatted readme in github can feel free to change this file over to the github readme and depreciate this file if desired.

\section{The Purpose of The Experimental Repo}
    
    The purpose of this repo to provide a (centralized) place where new features developed by one of the major Ximera server hosts can be submitted for all the other major server hosts to try and deploy to see if the new code will deploy on their server. This will help us centralize the testing and submission process to keep all the servers on the same core code, despite their individual customizations and changes. \\
    
    Importantly, this is not intended for \textbf{content} submissions, but rather for core code type changes, primarily to the Ximera latex package in some form. This will also provide a place to report bugs and resolve conflicts for individual new features before they are promoted to core code features.
    
\section{Different Types of Code Submissions}
    
    There are two primary types of code submissions currently expected for this repo - core file submissions and secondary file submissions. 
    
    \subsection{Core File Submissions}
        
        These are submissions for the core file types - the \verb|ximera.cls|, \verb|\ximera.4ht|, and/or \verb|xourse.cls| files. These changes should ideally be submitted in the form of modified \verb|.dtx| files whenever possible, to allow the other server hosts to build the core files on their own systems to make sure the changes generated by those dtx files don't cause syntax clashes with any of their customized dtx file contents. \\
        It is \textit{also} suggested that submissions include any of the core files that were modified so that those can be easily dropped into the \verb|.ximera| folder for fast-testing.\\
        So, for example, if Jason at UF created a way to implement problem numbers online, he would likely submit a modified \verb|problems.dtx| file (and possibly other dtx files). He would (ideally) \textit{also} submit the modified \verb|ximera.cls| and \verb|ximera.4ht| so that other server hosts could easily drop the \verb|ximera.cls| and \verb|ximera.4ht| files into the \verb|.ximera| folder and deploy a provided test file to verify the functionality of the new feature on their system. (More about the actual workflow below).
    
    \subsection{Secondary Files}
        
        Due to the current server setup, there are a very few types of new features that may be implemented that do not actually involve editing the core files, but rather are better tested using stand-alone example ximera assignments. The obvious example of this would be new validator code. These should be submitted via an example assignment, so that the other server hosts can easily copy the example file into their own system and deploy to test the features. (Again, more on the specific workflow below.)


\section{Intended Workflow}
    
    The intended workflow is designed to allow each major server host to submit new material for the others to test, while keeping track of what has - and hasn't - been tested by whom. 
    
    This repo consists of the core server hosts (referred to hereafter as the ``testing group'' for simplicity) that comprise most of the major servers that host Ximera. This list can be updated as more people begin hosting Ximera servers, and each server should have its own contribution folder, both as a way to recognize their membership in the group, as well as to give them a place to contribute new features.

    Inside each folder should be a contact/info document that lists a reliable way to contact someone for questions around their submissions, although ideally non-trivial problems with implementing a new feature can be reported using the issues tab in github.\\
    
    Any submission for testing/consideration from a server host would thus be placed inside their relevant folder, inside a dedicated subfolder for that feature. The expected contents will be described in a subsection below, but this is how we can handle tracking different features in parallel, even from the same contributor.\\
    
    Finally, alongside the other necessary files or testing, should be a text document for people that have signed off on successfully implementing the feature on their own servers. Once that document is signed by all the members of the testing group, that will flag the feature for promotion to the core code base, at which point the relevant modifications to the dtx files will be made for the release version of ximera and the feature will be added. This will also help us keep track of who has, and importantly who hasn't, actually tested any given feature still flagged as experimental.\\
    
    \subsection{Current Members of the Testing Group}
    
        The following are the current members of the testing group. This isn't intended to be some kind of ``elite'' list, but rather this list reflects all the servers (that we are aware of) interested in actively doing this kind of development work. If you want to be put on the list and become an active developer, you should contact Bart about joining.
        \begin{enumerate}
            \item OSU (Managed By: Bart Snapp/Jim Fowler)
            \item UF (Managed By: Jason Nowell)
            \item Kuleuven (Managed By: Wim Obbels)
        \end{enumerate}
    
    \subsection{Folder Contents}
        
        Each member of the testing group should have a folder at the base level of the repo that is named for their institution, e.g. ``UF Submissions'', ``OSU Submission'', etc. \\
        
        Inside this folder should be a file named \verb|contact.txt| that has some information about the person representing that institution for the development. Keep in mind that, by design, we don't want to promote experimental features to core features until \textit{everyone} has signed off on the feature, which means that the representative will need to be active enough to either check the repo regularly for new features to test, or at least to respond to requests to do so by other members that reach out via their contact information. Thus it is important that the representative include some way (eg an active email address) to contact them with requests in the \verb|contact.txt| file.
        
        This is the only required contents of each submission folder. The only other thing that should be in the base level of the submission folder should be other folders - each one representing a new feature submitted for testing.
        
    \subsection{Subfolder Contents for Submitted Features}
        
        When a testing group member wants to submit a new feature for testing, they should make a new folder in their submission folder for that feature. It is important to try - within reason - to provide each new submitted feature its own folder for testing - to minimize any kind of cross contamination during the testing phase for other group members.
        
        Inside this subfolder there should be any/all of the modified files that implement the feature, as well as examples, explanation, and confirmation files. This is likely to include:
        \begin{itemize}
            \item \texttt{.dtx} files that implement the feature into the core code base (once it has been promoted to a core feature) [Note that not all new features will have these files - e.g. new validators)
            \item A compiled version of the \texttt{ximera.cls}, \texttt{ximera.4ht}, \texttt{xourse.cls}, and/or \texttt{xourse.4ht} so that other test members can drop these into their \texttt{.ximera} folder for a quick-test
            \item A Ximera assignment (i.e. a xourse and corresponding ximera document class files) that implement the new feature fully. (See below)
            \item A file named \texttt{confirmed.txt} which will contain the list of who has tested and confirmed-working (i.e. ``green-lit'') the feature for adoption on their own servers.
            \item If the code that implements the new feature involved editing existing code, that may warrant including a \texttt{patchInfo.txt} file. (described below)
        \end{itemize}
        
        \subsubsection{Assignments for New Features}
            
            Along with the code that implements new features, submissions should include ximera assignments that \textit{completely} demonstrate the feature being proposed. This means that, for features that have a number of options (e.g. something like the answer command, which comes with a horde of optional arguments and key-value pairs) the assignment should include examples of each option individually, and ideally any combinations that have the potential to cause conflicts. This allows the other group members to simply drag-and-drop the tex files (a xourse file and at least one ximera file) into their testing folder and deploy it, without needing to write their own examples. \\
        
        \subsubsection{patchInfo.txt}
            
            Sometimes implementing a new feature involves patching existing code in very significant or important ways. As a result, this may create conflicts with features or code currently being developed at another server, but in very surprising and difficult to trace ways. Thus, at the contributor's discretion, they may include a \verb|patchInfo.txt| file that details specific aspects to how the new feature patches or utilizes current core code in highly unusual or unexpected ways that have potential to cause issues, but might be incredibly difficult to trace without guidance. This is entirely optional and can be addressed when issues are reported however.
            
\section{Testing Content}
    
    Test group members should be able to pull the repo to get a full download of all the currently proposed new features from everyone. They can then pull the contents of a given new feature subfolder and drop the files into the appropriate locations (any cls or 4ht files would go into the \verb|.ximera| folder, and the assignment and xourse files would go to the normal repo section). They can then deploy and see if the feature implements correctly for them. \\
    
    If everything works perfectly, the tester can then sign the \verb|confirmed.txt| file and then push the \verb|confirmed.txt| file back to the github repo - to signify that the test group member has confirmed the new feature as working on their server setup and authorizes it for promotion to core code.\\
    
    If there are problems, or if there are considerations that the tester wants to address (aside from just not compiling or rendering correctly etc) before signing off on promoting the new feature to core-code status, they can either address it via contacting the person that submitted the code (via the contact info from the \verb|contact.txt| file) for very minor issues/concerns, or (ideally) they can submit an ``issue'' on Github about the issue/concern they have. They should make sure to indicate in the issue title the feature and who submitted the feature, that they are addressing - to make it easy for people to search and scan through the issues section to find relevant topics that need their attention.\\






\end{document}
