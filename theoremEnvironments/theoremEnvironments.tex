\documentclass{ximera}
\outcome{Theorem environments.}
\newcommand{\dfn}{\textbf}
\renewcommand{\vec}[1]{{\overset{\boldsymbol{\rightharpoonup}}{\mathbf{#1}}}\hspace{0in}}
%% Simple horiz vectors
\renewcommand{\vector}[1]{\left\langle #1\right\rangle}
\newcommand{\arrowvec}[1]{{\overset{\rightharpoonup}{#1}}}
\newcommand{\R}{\mathbb{R}}
\newcommand{\transpose}{\intercal}






\usepackage{mdframed} % For framing content
%\usepackage{ifthen}   % For conditional statements

% Define the 'concept' environment with an optional header
\newenvironment{concept}[1][]{%
  \begin{mdframed}[linecolor=black, linewidth=2pt, innertopmargin=5pt, innerbottommargin=5pt, skipabove=12pt, skipbelow=12pt]%
    \noindent\large\textbf{#1}\normalsize%
}{%
  \end{mdframed}%
}











\colorlet{textColor}{black}
\colorlet{background}{white}
\colorlet{penColor}{blue!50!black} % Color of a curve in a plot
\colorlet{penColor2}{red!50!black}% Color of a curve in a plot
\colorlet{penColor3}{red!50!blue} % Color of a curve in a plot
\colorlet{penColor4}{green!50!black} % Color of a curve in a plot
\colorlet{penColor5}{orange!80!black} % Color of a curve in a plot
\colorlet{penColor6}{yellow!70!black} % Color of a curve in a plot
\colorlet{fill1}{penColor!20} % Color of fill in a plot
\colorlet{fill2}{penColor2!20} % Color of fill in a plot
\colorlet{fillp}{fill1} % Color of positive area
\colorlet{filln}{penColor2!20} % Color of negative area
\colorlet{fill3}{penColor3!20} % Fill
\colorlet{fill4}{penColor4!20} % Fill
\colorlet{fill5}{penColor5!20} % Fill
\colorlet{gridColor}{gray!50} % Color of grid in a plot



\usepackage[utf8]{inputenc}

% 201908/202301: PAS OP: babel en doclicense lijken problemen te veroorzaken in .jax bestand
% (wegens syntax error met toegevoegde \newcommands ...)
\pdfOnly{
    \usepackage[type={CC},modifier={by-nc-sa},version={4.0}]{doclicense}
    \usepackage[dutch]{babel}
}

%
% define softer blue/red/green, use KU Leuven base colors for blue (and dark orange for red ?)
%
% todo: rather redefine blue/red/green ...?
%\definecolor{xmblue}{rgb}{0.01, 0.31, 0.59}
%\definecolor{xmred}{rgb}{0.89, 0.02, 0.17}
\definecolor{xmdarkblue}{rgb}{0.122, 0.671, 0.835}   % KU Leuven Blauw
\definecolor{xmblue}{rgb}{0.114, 0.553, 0.69}        % KU Leuven Blauw
\definecolor{xmgreen}{rgb}{0.13, 0.55, 0.13}         % No KULeuven variant for green found ...

\definecolor{xmaccent}{rgb}{0.867, 0.541, 0.18}      % KU Leuven Accent (orange ...)
\definecolor{kuaccent}{rgb}{0.867, 0.541, 0.18}      % KU Leuven Accent (orange ...)

\colorlet{xmred}{xmaccent!50!black}                  % Darker version of KU Leuven Accent

\providecommand{\blue}[1]{{\color{blue}#1}}    
\providecommand{\red}[1]{{\color{red}#1}}


\providecommand{\isEn}{true}   % set to English

%
\ifdefined\isEn
 \newcommand{\nlen}[2]{#2}
 \newcommand{\nlentext}[2]{\text{#2}}
 \newcommand{\nlentextbf}[2]{\textbf{#2}}
\else
 \newcommand{\nlen}[2]{#1}
 \newcommand{\nlentext}[2]{\text{#1}}
 \newcommand{\nlentextbf}[2]{\textbf{#1}}
\fi

\newcommand\xmsection\subsection
\newcommand\xmsubsection\subsubsection

% Aanpassen printversie
%  (hier gedefinieerd, zodat ze in xourse kunnen worden gezet/overschreven)

\providebool{printbasicversion}
\providebool{printextendedversion}   % not properly implemented
\providebool{printfullversion}       % presumably print everything (debug/auteur)


\providebool{parttoc}
\providebool{printpartfrontpage}
\providebool{printactivitytitle}
\providebool{printactivityqrcode}
\providebool{printactivityurl}
\providebool{printcontinuouspagenumbers}
\providebool{numberactivitiesbysubpart}
\providebool{addtitlenumber}
\providebool{addsectiontitlenumber}
\addtitlenumbertrue
\addsectiontitlenumbertrue


% Commando om de printstyle toe te voegen in ximera's. Zorgt ervoor dat er geen problemen zijn als je de xourses compileert
% hack om onhandige relative paden in TeX te omzeilen
% should work both in xourse and ximera (pre-112022 only in ximera; thus obsoletes adhoc setup in xourses)
% loads global.sty if present (cfr global.css for online settings!)
% use global.sty to overwrite settings in printstyle.sty ...
\newcommand{\addPrintStyle}[1]{
\iftikzexport\else   % only in PDF
  \makeatletter
  \ifx\@onlypreamble\@notprerr\else   % ONLY if in tex-preamble   (and e.g. not when included from xourse)
    \typeout{Loading printstyle}   % logging
    \usepackage{#1/printstyle} % mag enkel geinclude worden als je die apart compileert
    \IfFileExists{#1/global.sty}{
        \typeout{Loading printstyle-folder #1/global.sty}   % logging
        \usepackage{#1/global}
        }{
        \typeout{Info: No extra #1/global.sty}   % logging
    }   % load global.sty if present
    \IfFileExists{global.sty}{
        \typeout{Loading local-folder global.sty (or TEXINPUTPATH..)}   % logging
        \usepackage{global}
    }{
        \typeout{Info: No folder/global.sty}   % logging
    }   % load global.sty if present
    \IfFileExists{\currfilebase.sty}
    {
        \typeout{Loading \currfilebase.sty}
        \input{\currfilebase.sty}
    }{
        \typeout{Info: No local \currfilebase.sty}
    }
    \fi
  \makeatother
\fi
}


\author{Bart Snapp \and Rodney Austin}

\title{Theorem-like environments}


\begin{document}
\begin{abstract}
  Examples of the theorem environments.
\end{abstract}
\maketitle


But, \ximera\ provides a number of theorm-like environments.

\begin{theorem}
 \lipsum[1][1-3]
 \begin{proof}
  \lipsum[1][1-3]
\[
\frac{\partial V}{\partial t} + \frac{1}{2} \sigma^2 S^2 \frac{\partial^2 V}{\partial S^2} + r S \frac{\partial V}{\partial S} - r V = \answer{0}
\]
\lipsum[1][1-3]
 \end{proof}
\end{theorem}


\begin{theorem}[My theorem]
  \lipsum[1][1-3]
  \begin{proof}
    \lipsum[1][1-3]
    \paragraph{Case 1:}
    \lipsum[1][1-2]
    \[
\begin{pmatrix}
x & 2x & 3x \\
4x & 5x & 6x \\
7x & 8x & 9x
\end{pmatrix}
\begin{pmatrix}
a \\
b \\
c
\end{pmatrix}
=
\begin{pmatrix}
\answer{ax + 2b x+ 3cx} \\
\answer{4ax + 5bx + 6cx} \\
\answer{7a x+ 8bx + 9cx}
\end{pmatrix}
\]
\lipsum[1][1-1]
\paragraph{Case2:}
\lipsum[1][1-3]


  \end{proof}
\end{theorem}

\begin{algorithm}
  \lipsum[1][1-3]
\end{algorithm}

\begin{axiom}
  \lipsum[1][1-3]
\end{axiom}

\begin{claim}
  \lipsum[1][1-3]
\end{claim}

\begin{conclusion}
  \lipsum[1][1-3]
\end{conclusion}

\begin{condition}
  \lipsum[1][1-3]
\end{condition}

\begin{conjecture}
  \lipsum[1][1-3]
\end{conjecture}

\begin{corollary}
  \lipsum[1][1-3]
\end{corollary}

\begin{criterion}
  \lipsum[1][1-3]
\end{criterion}

\begin{definition}
  \lipsum[1][1-3]
\end{definition}

\begin{example}
  \lipsum[1][1-3]
  \begin{solution}
    \lipsum[1][1-3]
    \begin{align*}
      ax^2 + bx + c &= 0 \\
      x^2 + \frac{b}{a}x + \frac{c}{a} &= 0 \\
      x^2 + \frac{b}{a}x &= -\frac{c}{a} \\
      x^2 + \frac{b}{a}x + \left(\frac{b}{2a}\right)^2 &= \left(\frac{b}{2a}\right)^2 - \frac{c}{a} \\
      \left(x + \frac{b}{2a}\right)^2 &= \frac{b^2 - 4ac}{4a^2} \\
      x + \frac{b}{2a} &= \pm \frac{\sqrt{\answer{b^2 - 4ac}}}{2a} \\
      x &= \frac{-b \pm \sqrt{b^2 - 4ac}}{2a}
      \end{align*}
    \lipsum[1][1-3]
  \end{solution}
\end{example}

\begin{explanation}
  \lipsum[1][1-3]
\end{explanation}

\begin{fact}
  \lipsum[1][1-3]
\end{fact}

\begin{formula}
  \lipsum[1][1-3]
\end{formula}

\begin{idea}
  \lipsum[1][1-3]
\end{idea}

\begin{lemma}
  \lipsum[1][1-3]
  \begin{proof}
    \lipsum[1][1-3]
  \end{proof}
\end{lemma}

\begin{model}
  \lipsum[1][1-3]
\end{model}

\begin{notation}
  \lipsum[1][1-3]
\end{notation}

\begin{observation}
  \lipsum[1][1-3]
\end{observation}

\begin{paradox}
  \lipsum[1][1-3]
  \begin{align*}
    e^x &= 1 + x + \frac{x^2}{2!} + \frac{x^3}{3!} + \cdots \\
    &= 1 + \answer{x} + \frac{\answer{x^2}}{2} + \frac{\answer{x^3}}{6} + \cdots
    \end{align*}
    \lipsum[1][1-3]
\end{paradox}

\begin{procedure}
  \lipsum[1][1-3]
\end{procedure}

\begin{proposition}
  \lipsum[1][1-3]
  \begin{proof}
    \lipsum[1][1-3]
  \end{proof}
\end{proposition}

\begin{remark}
  \lipsum[1][1-3]
\end{remark}

\begin{summary}
  \lipsum[1][1-3]
\end{summary}

\begin{template}
  \lipsum[1][1-3]
\end{template}

\begin{warning}
  \lipsum[1][1-3]
\end{warning}


\end{document}
