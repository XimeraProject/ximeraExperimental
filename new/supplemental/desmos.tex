\documentclass{ximera}
\title{Desmos Embeddings}


\begin{document}
\begin{abstract}
    How to embed Desmos Lessons
\end{abstract}
\maketitle

    \subsection*{How's it work?}
    
        You can embed Desmos lessons into Ximera easily, much like you can with YouTube, using the \verb|\desmos| command. Similarly to the youtube command, Desmos will take in a portion of the url (once the Desmos lesson is marked as embeddable) and configure everything else for you. By way of example, I have a lesson I give student to be able to play around with and understand translations and transformations here: 
        
        Thus if you wanted to share a Desmos lesson at \url{https://www.desmos.com/calculator/ryvd3xeltm}. To embed this, all we need is the string after the last slash, so we would use the command: \\ \verb|\desmos{ryvd3xeltm}{}{500}|\footnote{the second argument is the width - which defaults to the width of the window. The third is the height, which usually defaults to something too small, so I typically use 500 or so.}. Which gives us:
    
            \desmos{ryvd3xeltm}{}{500}
    
    \subsection*{Validation}
        Playing around with Desmos lessons has no natural ``completion'' metric, so it is currently ungraded - meaning that student interaction doesn't count toward progress on the page in any way.
    
        
    \subsection*{Optional Arguments}
        There are currently no optional arguments for Desmos.
    
    \subsection*{(Additional) Examples}
    
        Do we need additional examples here? Maybe some examples from the best practices below?
        
    \subsection*{Best Practices}
    
        \subsubsection*{Use Desmos as a ``playground''}
            Since embedding desmos doesn't allow for a graded evaluation of how the student interacts with the embedding (see potential problems and pitfalls below) there isn't a way to \textit{directly}\footnote{But you can indirectly, see below} assess the student's interaction with the embedded Desmos lesson. Nonetheless, Desmos is still a great tool for students to build intuition around an idea. For example, you could build a Desmos lesson that has a graph of a sign curve, and then sliders for phase shift, vertical shift, frequency, and/or amplitude. The student can then play with the sliders to watch how the graph warps and shifts as the different values are manipulated. This doesn't culminate in a piece of knowledge you can assess directly, but it helps build understanding and visualizing the impact of these features in a way that writing out a bunch of graphs simply doesn't capture.
        
        \subsubsection*{Indirectly Assessing Desmos Interaction}
            There are a couple ways you could use Desmos as a kind of ``random graphing'' tool, and/or assess the student's correct usage of a graph. The first is to supply a number of parameters to the student (possibly randomly generated) which they can plug into an embedded Desmos graph's prompts, then ask the student questions about the resulting Desmos graph. This does rely on the student correctly entering in parameters themselves to generate the graph, but this can be made simple enough that there should be a low rate of issues/frustrations around setting up the graph.
            
        \subsubsection*{Indirectly Assessing Desmos Interaction pt 2}
            Another way this can be used, is to provide a graph or even a description of a graph, and then a Desmos embedding with some number of editable parameters. The student would then be asked to tinker with the parameters until they match the provided graph or description. Then the student would have to provide parameters in answer boxes to confirm they correctly manipulated the Desmos graph to the desired result. 
    
    \subsection*{Potential Pitfalls and Problems}
        \subsubsection*{No grade potential}
            Since students can't earn points from it, it can sometimes be difficult to motivate them to interact with a Desmos lesson. One option is to make followup problems that require some kind of manipulation within your Desmos lesson in order to be able to answer them, but finding a motivational method is up to the creativity of the author.
    
\end{document}