\documentclass{ximera}
\title{Xourse and Ximera File Layout}


\begin{document}
\begin{abstract}
    How to write and layout the xourse and Ximera documentClass file to build an assignment.
\end{abstract}
\maketitle

There are two document classes for making a Ximera assignment. The
\verb|xourse| type is the file that turns into the page with all the tiles
on it that students click on to load the page with the actual content
on it. The tiles with the actual content are \verb|ximera|
documentClass. We discuss the xourse documentClass here, whereas the
rest of this documentation is largely for the content that goes into
the Ximera documentClass files.


\section{Titles and abstracts}

Titles and abstracts help users find your content. Typically,
activites need titles and an abstract.  For example:

\begin{verbatim}
\documentclass{ximera}
\title{Title of your activity}
\author{Your name here}
\begin{document}
\begin{abstract}
  A one-sentence description of the activity.
\end{abstract}
\maketitle
\end{document}
\end{verbatim}
makes a blank Ximera document with the title ``Title of your
activity'' The \verb|\title| command is what generates the title of
the tile, and the subtitle for a tile is generated using the
`abstract' environment - the content of that environment is what gets
put as the subtitle/description. The title command should be placed
before the \verb|\begin{document}| command, whereas the abstract
should be placed \textit{after} the \verb|\begin{document}|
command. Finally, in order to make the title and subtitle to appear on
the tile, you must use the \verb|\maketitle| command. This is
especially important, since otherwise the tile is blank and remarkably
difficult to see on some devices.



\section{Title styles}


To help make a book, Ximera provides syle commands for the titles of
the activities. These should be used in the \verb|xourse| file.

\begin{verbatim}
\documentclass{xourse}
\title{Title of your activity}
\author{Your name here}
\begin{document}
\begin{abstract}
  A one-sentence description of the xourse (optional).
\end{abstract}
\maketitle

\chapterstyle
\activity{path-to-someActivity.tex}
\sectionstyle
\activity{path-to-someOtherActivity.tex}

\end{document}
\end{verbatim}



    
\section{Xourse File Content}



The \verb|xourse| documentClass contains the paths to the course
files, and the nature of those tiles is determined by the command used
to load them. There are two options, the \verb|\activity| command
creates a full sized tile, with a title and description displayed, and
the \verb|\practice| command creates a thin tile that shows only the
progress bar, without a title or description. These commands are
effectively an \verb|\include| command, so they should point to a
Ximera documentClass file (note that the file itself doesn't need any
special configuration to be an activity versus practice).



You can also use the \verb|\part| command to make a darker tile that
is not clickable, which can be useful to divide up your
content. Moreover, there is also a \verb|\chapterstyle| and
\verb|\sectionstyle| command, which make all activities after those
commands have slightly different coloring - e.g. chapterstyle makes
the tile slightly darker than section style, which is the
default. Thus you could have a xourse file that looks something like:
    
\begin{verbatim}
\documentClass{xourse}
\newcommand{\dfn}{\textbf}
\renewcommand{\vec}[1]{{\overset{\boldsymbol{\rightharpoonup}}{\mathbf{#1}}}\hspace{0in}}
%% Simple horiz vectors
\renewcommand{\vector}[1]{\left\langle #1\right\rangle}
\newcommand{\arrowvec}[1]{{\overset{\rightharpoonup}{#1}}}
\newcommand{\R}{\mathbb{R}}
\newcommand{\transpose}{\intercal}






\usepackage{mdframed} % For framing content
%\usepackage{ifthen}   % For conditional statements

% Define the 'concept' environment with an optional header
\newenvironment{concept}[1][]{%
  \begin{mdframed}[linecolor=black, linewidth=2pt, innertopmargin=5pt, innerbottommargin=5pt, skipabove=12pt, skipbelow=12pt]%
    \noindent\large\textbf{#1}\normalsize%
}{%
  \end{mdframed}%
}











\colorlet{textColor}{black}
\colorlet{background}{white}
\colorlet{penColor}{blue!50!black} % Color of a curve in a plot
\colorlet{penColor2}{red!50!black}% Color of a curve in a plot
\colorlet{penColor3}{red!50!blue} % Color of a curve in a plot
\colorlet{penColor4}{green!50!black} % Color of a curve in a plot
\colorlet{penColor5}{orange!80!black} % Color of a curve in a plot
\colorlet{penColor6}{yellow!70!black} % Color of a curve in a plot
\colorlet{fill1}{penColor!20} % Color of fill in a plot
\colorlet{fill2}{penColor2!20} % Color of fill in a plot
\colorlet{fillp}{fill1} % Color of positive area
\colorlet{filln}{penColor2!20} % Color of negative area
\colorlet{fill3}{penColor3!20} % Fill
\colorlet{fill4}{penColor4!20} % Fill
\colorlet{fill5}{penColor5!20} % Fill
\colorlet{gridColor}{gray!50} % Color of grid in a plot



\usepackage[utf8]{inputenc}

% 201908/202301: PAS OP: babel en doclicense lijken problemen te veroorzaken in .jax bestand
% (wegens syntax error met toegevoegde \newcommands ...)
\pdfOnly{
    \usepackage[type={CC},modifier={by-nc-sa},version={4.0}]{doclicense}
    \usepackage[dutch]{babel}
}

%
% define softer blue/red/green, use KU Leuven base colors for blue (and dark orange for red ?)
%
% todo: rather redefine blue/red/green ...?
%\definecolor{xmblue}{rgb}{0.01, 0.31, 0.59}
%\definecolor{xmred}{rgb}{0.89, 0.02, 0.17}
\definecolor{xmdarkblue}{rgb}{0.122, 0.671, 0.835}   % KU Leuven Blauw
\definecolor{xmblue}{rgb}{0.114, 0.553, 0.69}        % KU Leuven Blauw
\definecolor{xmgreen}{rgb}{0.13, 0.55, 0.13}         % No KULeuven variant for green found ...

\definecolor{xmaccent}{rgb}{0.867, 0.541, 0.18}      % KU Leuven Accent (orange ...)
\definecolor{kuaccent}{rgb}{0.867, 0.541, 0.18}      % KU Leuven Accent (orange ...)

\colorlet{xmred}{xmaccent!50!black}                  % Darker version of KU Leuven Accent

\providecommand{\blue}[1]{{\color{blue}#1}}    
\providecommand{\red}[1]{{\color{red}#1}}


\providecommand{\isEn}{true}   % set to English

%
\ifdefined\isEn
 \newcommand{\nlen}[2]{#2}
 \newcommand{\nlentext}[2]{\text{#2}}
 \newcommand{\nlentextbf}[2]{\textbf{#2}}
\else
 \newcommand{\nlen}[2]{#1}
 \newcommand{\nlentext}[2]{\text{#1}}
 \newcommand{\nlentextbf}[2]{\textbf{#1}}
\fi

\newcommand\xmsection\subsection
\newcommand\xmsubsection\subsubsection

% Aanpassen printversie
%  (hier gedefinieerd, zodat ze in xourse kunnen worden gezet/overschreven)

\providebool{printbasicversion}
\providebool{printextendedversion}   % not properly implemented
\providebool{printfullversion}       % presumably print everything (debug/auteur)


\providebool{parttoc}
\providebool{printpartfrontpage}
\providebool{printactivitytitle}
\providebool{printactivityqrcode}
\providebool{printactivityurl}
\providebool{printcontinuouspagenumbers}
\providebool{numberactivitiesbysubpart}
\providebool{addtitlenumber}
\providebool{addsectiontitlenumber}
\addtitlenumbertrue
\addsectiontitlenumbertrue


% Commando om de printstyle toe te voegen in ximera's. Zorgt ervoor dat er geen problemen zijn als je de xourses compileert
% hack om onhandige relative paden in TeX te omzeilen
% should work both in xourse and ximera (pre-112022 only in ximera; thus obsoletes adhoc setup in xourses)
% loads global.sty if present (cfr global.css for online settings!)
% use global.sty to overwrite settings in printstyle.sty ...
\newcommand{\addPrintStyle}[1]{
\iftikzexport\else   % only in PDF
  \makeatletter
  \ifx\@onlypreamble\@notprerr\else   % ONLY if in tex-preamble   (and e.g. not when included from xourse)
    \typeout{Loading printstyle}   % logging
    \usepackage{#1/printstyle} % mag enkel geinclude worden als je die apart compileert
    \IfFileExists{#1/global.sty}{
        \typeout{Loading printstyle-folder #1/global.sty}   % logging
        \usepackage{#1/global}
        }{
        \typeout{Info: No extra #1/global.sty}   % logging
    }   % load global.sty if present
    \IfFileExists{global.sty}{
        \typeout{Loading local-folder global.sty (or TEXINPUTPATH..)}   % logging
        \usepackage{global}
    }{
        \typeout{Info: No folder/global.sty}   % logging
    }   % load global.sty if present
    \IfFileExists{\currfilebase.sty}
    {
        \typeout{Loading \currfilebase.sty}
        \input{\currfilebase.sty}
    }{
        \typeout{Info: No local \currfilebase.sty}
    }
    \fi
  \makeatother
\fi
}


\title{Example Assignment}% Title of the tile
\begin{document}
\begin{abstract}% Write the description
This is an example assignment containing a few tiles of problems.
\end{abstract}
\maketitle% Make sure to display the title and description

\part{First Assignment Type}% Declare first assignment block via a large darker tile
\activity{folderOne/assignmentOne}% Load assignmentOne.tex in the subfolder folderOne
\practice{folderOne/practice/practiceOne}% Make a thin practice tile out of the file ``practiceOne'' contained in the subfolder ``practice'' in the subfolder ``folderTwo''.

\part{Second Assignment Type}% Declare second assignment block via a large darker tile
\chapterstyle% Acitivities are going to be shaded darker until we reset to styletype
\activity{folderTwo/assignmentOneLeadIn}% a lead-in for the assignments, so it gets special coloring
\sectionstyle% Return to normal tile coloring.
\activity{folderTwo/assignmentOne}% Load assignmentOne.tex from the subfolder: folderTwo
\practice{folderTwo/practice/practiceOne}% Make a thin practice tile out of the file contained in the subfolder ``practice'' in the subfolder ``folderTwo''.
\end{document}
\end{verbatim}

Note that when online at (BADBAD Give link to official docs!)you can
go to the xourse file for this documentation by going back to the page
with all the tiles, then append ``.tex'' to the end of the url and see
the xourse file that generated this page to get a real-life example.



\section{Content in the preamble}


Ximera documents are meant to be compilable either indvidually or as
part of a Xourse. In order to have consistent behavior between Ximera class files and Xourse files, and to avoid option classes, the best-practice is to create a separate preamble file. 
% This `magic' is achieved by `turning off'
% \verb|\begin{document}| and \verb|\end{document}| and doing some other
% things. This can cause code between \verb|\end{document}| and
% \verb|\begin{document}| can be exposed when used in a Xourse
% file. This is not a big issue, but it means that the preamble of a
When compiling the Xourse document, the Ximera document can only contain objects like

\begin{verbatim}
\documentclass...
\title...
\author...
\date...
\input... ????
\end{verbatim}



% to input into both the Ximera document and the Xourse
% and input that file wherever you need a preamble - including the
% xourse file and any Ximera files.  It's important to know that any
% preamble you include in any Ximera documentClass document will be
% obliterated as part of the publishing process - instead it will pull
% the preamble from the xourse documentClass, so it's important to make
% sure you put everything for every preamble you need into the one file
% that is loaded in the xourse documentClass file - although in order to
% compile via pdflatex locally, you will also want to input that same
% preamble file into most of your Ximera documentClass files as well.
    
%     The only exception to the previous preamble being obliterated is the title command and its content.
    
 
    

\end{document}







