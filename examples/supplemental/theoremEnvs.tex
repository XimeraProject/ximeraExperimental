\documentclass{ximera}
\outcome{Theorem environments.}
\newcommand{\dfn}{\textbf}
\renewcommand{\vec}[1]{{\overset{\boldsymbol{\rightharpoonup}}{\mathbf{#1}}}\hspace{0in}}
%% Simple horiz vectors
\renewcommand{\vector}[1]{\left\langle #1\right\rangle}
\newcommand{\arrowvec}[1]{{\overset{\rightharpoonup}{#1}}}
\newcommand{\R}{\mathbb{R}}
\newcommand{\transpose}{\intercal}






\usepackage{mdframed} % For framing content
%\usepackage{ifthen}   % For conditional statements

% Define the 'concept' environment with an optional header
\newenvironment{concept}[1][]{%
  \begin{mdframed}[linecolor=black, linewidth=2pt, innertopmargin=5pt, innerbottommargin=5pt, skipabove=12pt, skipbelow=12pt]%
    \noindent\large\textbf{#1}\normalsize%
}{%
  \end{mdframed}%
}











\colorlet{textColor}{black}
\colorlet{background}{white}
\colorlet{penColor}{blue!50!black} % Color of a curve in a plot
\colorlet{penColor2}{red!50!black}% Color of a curve in a plot
\colorlet{penColor3}{red!50!blue} % Color of a curve in a plot
\colorlet{penColor4}{green!50!black} % Color of a curve in a plot
\colorlet{penColor5}{orange!80!black} % Color of a curve in a plot
\colorlet{penColor6}{yellow!70!black} % Color of a curve in a plot
\colorlet{fill1}{penColor!20} % Color of fill in a plot
\colorlet{fill2}{penColor2!20} % Color of fill in a plot
\colorlet{fillp}{fill1} % Color of positive area
\colorlet{filln}{penColor2!20} % Color of negative area
\colorlet{fill3}{penColor3!20} % Fill
\colorlet{fill4}{penColor4!20} % Fill
\colorlet{fill5}{penColor5!20} % Fill
\colorlet{gridColor}{gray!50} % Color of grid in a plot



\usepackage[utf8]{inputenc}

% 201908/202301: PAS OP: babel en doclicense lijken problemen te veroorzaken in .jax bestand
% (wegens syntax error met toegevoegde \newcommands ...)
\pdfOnly{
    \usepackage[type={CC},modifier={by-nc-sa},version={4.0}]{doclicense}
    \usepackage[dutch]{babel}
}

%
% define softer blue/red/green, use KU Leuven base colors for blue (and dark orange for red ?)
%
% todo: rather redefine blue/red/green ...?
%\definecolor{xmblue}{rgb}{0.01, 0.31, 0.59}
%\definecolor{xmred}{rgb}{0.89, 0.02, 0.17}
\definecolor{xmdarkblue}{rgb}{0.122, 0.671, 0.835}   % KU Leuven Blauw
\definecolor{xmblue}{rgb}{0.114, 0.553, 0.69}        % KU Leuven Blauw
\definecolor{xmgreen}{rgb}{0.13, 0.55, 0.13}         % No KULeuven variant for green found ...

\definecolor{xmaccent}{rgb}{0.867, 0.541, 0.18}      % KU Leuven Accent (orange ...)
\definecolor{kuaccent}{rgb}{0.867, 0.541, 0.18}      % KU Leuven Accent (orange ...)

\colorlet{xmred}{xmaccent!50!black}                  % Darker version of KU Leuven Accent

\providecommand{\blue}[1]{{\color{blue}#1}}    
\providecommand{\red}[1]{{\color{red}#1}}


\providecommand{\isEn}{true}   % set to English

%
\ifdefined\isEn
 \newcommand{\nlen}[2]{#2}
 \newcommand{\nlentext}[2]{\text{#2}}
 \newcommand{\nlentextbf}[2]{\textbf{#2}}
\else
 \newcommand{\nlen}[2]{#1}
 \newcommand{\nlentext}[2]{\text{#1}}
 \newcommand{\nlentextbf}[2]{\textbf{#1}}
\fi

\newcommand\xmsection\subsection
\newcommand\xmsubsection\subsubsection

% Aanpassen printversie
%  (hier gedefinieerd, zodat ze in xourse kunnen worden gezet/overschreven)

\providebool{printbasicversion}
\providebool{printextendedversion}   % not properly implemented
\providebool{printfullversion}       % presumably print everything (debug/auteur)


\providebool{parttoc}
\providebool{printpartfrontpage}
\providebool{printactivitytitle}
\providebool{printactivityqrcode}
\providebool{printactivityurl}
\providebool{printcontinuouspagenumbers}
\providebool{numberactivitiesbysubpart}
\providebool{addtitlenumber}
\providebool{addsectiontitlenumber}
\addtitlenumbertrue
\addsectiontitlenumbertrue


% Commando om de printstyle toe te voegen in ximera's. Zorgt ervoor dat er geen problemen zijn als je de xourses compileert
% hack om onhandige relative paden in TeX te omzeilen
% should work both in xourse and ximera (pre-112022 only in ximera; thus obsoletes adhoc setup in xourses)
% loads global.sty if present (cfr global.css for online settings!)
% use global.sty to overwrite settings in printstyle.sty ...
\newcommand{\addPrintStyle}[1]{
\iftikzexport\else   % only in PDF
  \makeatletter
  \ifx\@onlypreamble\@notprerr\else   % ONLY if in tex-preamble   (and e.g. not when included from xourse)
    \typeout{Loading printstyle}   % logging
    \usepackage{#1/printstyle} % mag enkel geinclude worden als je die apart compileert
    \IfFileExists{#1/global.sty}{
        \typeout{Loading printstyle-folder #1/global.sty}   % logging
        \usepackage{#1/global}
        }{
        \typeout{Info: No extra #1/global.sty}   % logging
    }   % load global.sty if present
    \IfFileExists{global.sty}{
        \typeout{Loading local-folder global.sty (or TEXINPUTPATH..)}   % logging
        \usepackage{global}
    }{
        \typeout{Info: No folder/global.sty}   % logging
    }   % load global.sty if present
    \IfFileExists{\currfilebase.sty}
    {
        \typeout{Loading \currfilebase.sty}
        \input{\currfilebase.sty}
    }{
        \typeout{Info: No local \currfilebase.sty}
    }
    \fi
  \makeatother
\fi
}


\author{Bart Snapp \and Rodney Austin}

\title{Theorem-like environments}


\begin{document}
\begin{abstract}
  Examples of the theorem environments.
\end{abstract}
\maketitle

%\section*{Theorem-Like Environments}
    \subsection*{How theorem-like environments work}
        Ximera provides a number of theorem-like environments by default. These are all functionally identical, the only difference is the name associated with them. It is important to note that each of these theorem-like environments use their own independent counter/numbering. For example, observe the numberings of the following theorems and corollaries:
        
        \begin{theorem}
            Here's the first theorem!
        \end{theorem}
        
        \begin{corollary}
            Here's the first corollary!
        \end{corollary}
    
        \begin{corollary}
            Here's the second corollary!
        \end{corollary}
    
        \begin{theorem}
            Here's the second theorem!
        \end{theorem}
        
        So the different environments will number themselves in order, but each tracks its own numbering independently.

    \subsection*{Optional Arguments}
        
        There are (currently) no Ximera-specific optional arguments - but Ximera does support the default optional arguments. For example, you can put an optional argument to provide a ``nickname'' for the theorem. For example:
        
        \begin{axiom}[Axiom of Choice]
            I'm not going to actually write the Axiom of Choice here - but only crazy people don't assume it. Or maybe non-analysts.
        \end{axiom}
        
        The above is generated with:
        \begin{verbatim}
        \begin{axiom}[Axiom of Choice]
            I'm not going to actually write the Axiom of Choice here - 
            but only crazy people don't assume it. Or maybe non-analysts.
        \end{axiom}        
        \end{verbatim}
        
        
    \subsection*{Examples}
        The easiest way to list the default theorem-like environments, is just to demonstrate them all - prepare for a massive example list!
        
        \begin{theorem}
            \lipsum[1][1-3]
        \end{theorem}
        
        
        \begin{theorem}[My theorem]
            \lipsum[1][1-3]
        \end{theorem}
        
        \begin{algorithm}
            \lipsum[1][1-3]
        \end{algorithm}
        
        \begin{axiom}
            \lipsum[1][1-3]
        \end{axiom}
        
        \begin{claim}
            \lipsum[1][1-3]
        \end{claim}
        
        \begin{conclusion}
            \lipsum[1][1-3]
        \end{conclusion}
        
        \begin{condition}
            \lipsum[1][1-3]
        \end{condition}
        
        \begin{conjecture}
            \lipsum[1][1-3]
        \end{conjecture}
        
        \begin{corollary}
            \lipsum[1][1-3]
        \end{corollary}
        
        \begin{criterion}
            \lipsum[1][1-3]
        \end{criterion}
        
        \begin{definition}
            \lipsum[1][1-3]
        \end{definition}
        
        \begin{example}
            \lipsum[1][1-3]
        \end{example}
        
        \begin{explanation}
            \lipsum[1][1-3]
        \end{explanation}
        
        \begin{fact}
            \lipsum[1][1-3]
        \end{fact}
        
        \begin{formula}
            \lipsum[1][1-3]
        \end{formula}
        
        \begin{idea}
            \lipsum[1][1-3]
        \end{idea}
        
        \begin{lemma}
            \lipsum[1][1-3]
        \end{lemma}
        
        \begin{model}
            \lipsum[1][1-3]
        \end{model}
        
        \begin{notation}
            \lipsum[1][1-3]
        \end{notation}
        
        \begin{observation}
            \lipsum[1][1-3]
        \end{observation}
        
        \begin{paradox}
            \lipsum[1][1-3]
        \end{paradox}
        
        \begin{procedure}
            \lipsum[1][1-3]
        \end{procedure}
        
        \begin{proposition}
            \lipsum[1][1-3]
        \end{proposition}
        
        \begin{remark}
            \lipsum[1][1-3]
        \end{remark}
        
        \begin{summary}
            \lipsum[1][1-3]
        \end{summary}
        
        \begin{template}
            \lipsum[1][1-3]
        \end{template}
        
        \begin{warning}
            \lipsum[1][1-3]
        \end{warning}

    \subsection*{Best Practices}
        
        
        
    \subsection*{Potential Pitfalls and Problems}
        \subsubsection*{Page Credit}
            Remember that environments are what generate page credit in Ximera. So if you include one of these environments, and don't have anything which requires student input inside that environment (such as a multiple choice environment or an answer command), it will still count as credit for the page, and immediately grant student credit for that environment as it will immediately be marked as ``completed'' (since there is nothing for the student to do). This can be a good thing - to use as partial credit for at least opening the assignment for instance. See the segment on Credit Allocation for more on this issue in general.
        \subsubsection*{Cross-Referencing}
            One of LaTeX's strong points is automatically numbering and cross-referencing content via a label and ref command. This still works in the generated pdf (at time of writing), but it does \textbf{not} work in any meaningful way online. In general, one is best served by using a hard-coded reference and possibly a link command to reference another part of the document, to make sure that it works online as well as online. See the advanced section on linking content for more on this. 
        
        
\end{document}
