%%
%% Generated by gpt_translate from test/questions.tex, on 2024-07-09 15:37:18 using model gpt-3.5-turbo-16k
%%

% GPT CHUNK%
\documentclass{ximera}
\newcommand{\dfn}{\textbf}
\renewcommand{\vec}[1]{{\overset{\boldsymbol{\rightharpoonup}}{\mathbf{#1}}}\hspace{0in}}
%% Simple horiz vectors
\renewcommand{\vector}[1]{\left\langle #1\right\rangle}
\newcommand{\arrowvec}[1]{{\overset{\rightharpoonup}{#1}}}
\newcommand{\R}{\mathbb{R}}
\newcommand{\transpose}{\intercal}






\usepackage{mdframed} % For framing content
%\usepackage{ifthen}   % For conditional statements

% Define the 'concept' environment with an optional header
\newenvironment{concept}[1][]{%
  \begin{mdframed}[linecolor=black, linewidth=2pt, innertopmargin=5pt, innerbottommargin=5pt, skipabove=12pt, skipbelow=12pt]%
    \noindent\large\textbf{#1}\normalsize%
}{%
  \end{mdframed}%
}











\colorlet{textColor}{black}
\colorlet{background}{white}
\colorlet{penColor}{blue!50!black} % Color of a curve in a plot
\colorlet{penColor2}{red!50!black}% Color of a curve in a plot
\colorlet{penColor3}{red!50!blue} % Color of a curve in a plot
\colorlet{penColor4}{green!50!black} % Color of a curve in a plot
\colorlet{penColor5}{orange!80!black} % Color of a curve in a plot
\colorlet{penColor6}{yellow!70!black} % Color of a curve in a plot
\colorlet{fill1}{penColor!20} % Color of fill in a plot
\colorlet{fill2}{penColor2!20} % Color of fill in a plot
\colorlet{fillp}{fill1} % Color of positive area
\colorlet{filln}{penColor2!20} % Color of negative area
\colorlet{fill3}{penColor3!20} % Fill
\colorlet{fill4}{penColor4!20} % Fill
\colorlet{fill5}{penColor5!20} % Fill
\colorlet{gridColor}{gray!50} % Color of grid in a plot



\usepackage[utf8]{inputenc}

% 201908/202301: PAS OP: babel en doclicense lijken problemen te veroorzaken in .jax bestand
% (wegens syntax error met toegevoegde \newcommands ...)
\pdfOnly{
    \usepackage[type={CC},modifier={by-nc-sa},version={4.0}]{doclicense}
    \usepackage[dutch]{babel}
}

%
% define softer blue/red/green, use KU Leuven base colors for blue (and dark orange for red ?)
%
% todo: rather redefine blue/red/green ...?
%\definecolor{xmblue}{rgb}{0.01, 0.31, 0.59}
%\definecolor{xmred}{rgb}{0.89, 0.02, 0.17}
\definecolor{xmdarkblue}{rgb}{0.122, 0.671, 0.835}   % KU Leuven Blauw
\definecolor{xmblue}{rgb}{0.114, 0.553, 0.69}        % KU Leuven Blauw
\definecolor{xmgreen}{rgb}{0.13, 0.55, 0.13}         % No KULeuven variant for green found ...

\definecolor{xmaccent}{rgb}{0.867, 0.541, 0.18}      % KU Leuven Accent (orange ...)
\definecolor{kuaccent}{rgb}{0.867, 0.541, 0.18}      % KU Leuven Accent (orange ...)

\colorlet{xmred}{xmaccent!50!black}                  % Darker version of KU Leuven Accent

\providecommand{\blue}[1]{{\color{blue}#1}}    
\providecommand{\red}[1]{{\color{red}#1}}


\providecommand{\isEn}{true}   % set to English

%
\ifdefined\isEn
 \newcommand{\nlen}[2]{#2}
 \newcommand{\nlentext}[2]{\text{#2}}
 \newcommand{\nlentextbf}[2]{\textbf{#2}}
\else
 \newcommand{\nlen}[2]{#1}
 \newcommand{\nlentext}[2]{\text{#1}}
 \newcommand{\nlentextbf}[2]{\textbf{#1}}
\fi

\newcommand\xmsection\subsection
\newcommand\xmsubsection\subsubsection

% Aanpassen printversie
%  (hier gedefinieerd, zodat ze in xourse kunnen worden gezet/overschreven)

\providebool{printbasicversion}
\providebool{printextendedversion}   % not properly implemented
\providebool{printfullversion}       % presumably print everything (debug/auteur)


\providebool{parttoc}
\providebool{printpartfrontpage}
\providebool{printactivitytitle}
\providebool{printactivityqrcode}
\providebool{printactivityurl}
\providebool{printcontinuouspagenumbers}
\providebool{numberactivitiesbysubpart}
\providebool{addtitlenumber}
\providebool{addsectiontitlenumber}
\addtitlenumbertrue
\addsectiontitlenumbertrue


% Commando om de printstyle toe te voegen in ximera's. Zorgt ervoor dat er geen problemen zijn als je de xourses compileert
% hack om onhandige relative paden in TeX te omzeilen
% should work both in xourse and ximera (pre-112022 only in ximera; thus obsoletes adhoc setup in xourses)
% loads global.sty if present (cfr global.css for online settings!)
% use global.sty to overwrite settings in printstyle.sty ...
\newcommand{\addPrintStyle}[1]{
\iftikzexport\else   % only in PDF
  \makeatletter
  \ifx\@onlypreamble\@notprerr\else   % ONLY if in tex-preamble   (and e.g. not when included from xourse)
    \typeout{Loading printstyle}   % logging
    \usepackage{#1/printstyle} % mag enkel geinclude worden als je die apart compileert
    \IfFileExists{#1/global.sty}{
        \typeout{Loading printstyle-folder #1/global.sty}   % logging
        \usepackage{#1/global}
        }{
        \typeout{Info: No extra #1/global.sty}   % logging
    }   % load global.sty if present
    \IfFileExists{global.sty}{
        \typeout{Loading local-folder global.sty (or TEXINPUTPATH..)}   % logging
        \usepackage{global}
    }{
        \typeout{Info: No folder/global.sty}   % logging
    }   % load global.sty if present
    \IfFileExists{\currfilebase.sty}
    {
        \typeout{Loading \currfilebase.sty}
        \input{\currfilebase.sty}
    }{
        \typeout{Info: No local \currfilebase.sty}
    }
    \fi
  \makeatother
\fi
}


\addPrintStyle{..}

\begin{document}
    \xmtitle{Questions}{Simple example of exercise and question}

    \begin{exercise} Solve the following questions correctly:
        \begin{question}
                $1+1 = \answer{2}$
        \end{question}    
        \begin{question}
            $1+2 = $ \wordChoice{   \choice{$2$}
                                    \choice[correct]{$3$}
                                    \choice{$4$}
                                }
            \begin{hint}
                By definition, $2 = 1+1$, and addition is associative.
            \end{hint}
            \begin{feedback}[correct] 
                Congratulations, you can already do arithmetic very well! Keep it up. 
            \end{feedback}         
            \begin{oplossing} It holds that
                \[ 1 + 2 = 1 + (1 + 1) = (1 + 1) + 1 = 2 + 1 = 3 \]
                because of the definition $2\perdef 1+1$, the associativity of addition,
                and finally the definition $3\perdef 2+1$. 
            \end{oplossing}
        \end{question}

        \begin{question}$x-4 = 0
            \iff x = \answer{4}$
        \end{question}
        \begin{question}$x-4 = 0
            \iff x = \answer[onlineshowanswerbutton]{4}$
        \end{question}
        \begin{question}$x-4 = 0
            \iff x = \answer[onlinenoinput]{4}$
        \end{question}
        \begin{question}$x^2-4 < 0
            \iff \answer[onlinenoinput]{ x \in \left]-2,2\right[}$
        \end{question}

    \end{exercise}
    \begin{exercise} Tests with answer and feedback

        \begin{question}Write out $3: \answer{drie}$  (using \verb|\answer{drie}|)
        \end{question}

        \begin{question}Write out $3: \answer[format=string]{drie}$ (using \verb|\answer[format=string]{drie}|)
        \end{question}

        \begin{question}Write out $2$: $\answer[id=ansr,format=string]{twee}$ (with tests for feedback)
            \begin{feedback}[correct]{Correct}\end{feedback}
            \begin{feedback}[incorrect]{Incorrect}\end{feedback}    % !!! THIS DOES NOT WORK !!!!
            \begin{feedback}[ansr.toLowerCase() === 'twee']{Correct (because Ximera compares case-insensitive)}\end{feedback}  % but this is equivalent to [correct] ...
            \begin{feedback}[ansr === 'twee']{Correct (according to === in JS)}\end{feedback}  % but this is equivalent to [correct] ...
            \ifonline{
            \begin{feedback}[(ansr.toLowerCase() === 'twee') && (ansr !== 'twee')]{Correct (because Ximera compares case-insensitive) }\end{feedback}   % and this works !!!
            }
            {
            \begin{feedback}[]{Correct (because Ximera compares case-insensitive) }\end{feedback}   % NO && in PDF ...?  
            }
            \begin{feedback}[ansr < 'twee']{Too small}\end{feedback} % this also works, but is probably not very useful 
            \begin{feedback}[ansr > 'twee']{Too big}\end{feedback}
        \end{question}

        \begin{question}
            $\frac{1}{3} =  \answer[tolerance=0.05]{0.33}$  

            Use \verb|\answer[tolerance=0.05]{0.33}| to tolerate rounding errors. You can be off by 0.05 in this case.
        \end{question}

        \begin{question}
            $\frac{1}{3} =  \answer[tolerance=0.05]{\frac{1}{3}}$  

            Use \verb|\answer[tolerance=0.05]{1/3}| to tolerate rounding errors. You can be off by 0.05 in this case.
        \end{question}

        % \begin{question}
        %     $\frac{1}{3} =  \answer[tolerance=0.05]{0.33}$  

        %     Use \verb|\answer[tolerance=0.05]{0.33}| to tolerate rounding errors. You can be off by 0.05 in this case.
        % \end{question}        
        % \begin{question}
        %     $\frac{1}{3} =  \answer[tolerance=0.05]{0,33}$  

        %     Use \verb|\answer[tolerance=0.05]{0.33}| to tolerate rounding errors. You can be off by 0.05 in this case.
        % \end{question}

        \begin{question} Write $x^2: x^{\answer{2}}$
        \end{question}


    \end{exercise}

    \begin{example} Explore which $x\in \R$ satisfy $\left|\dfrac{2x}{3}\right|<1$.

        \begin{basicSkip}
        Solution set: $\answer[onlinenoinput]{\displaystyle
            V=\left]-\frac{3}{2},\frac{3}{2}\right[}$
        \end{basicSkip}
        \begin{oplossing}
            None
        \end{oplossing}
    \end{example}

\end{document}