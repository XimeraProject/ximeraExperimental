%%
%% Generated by gpt_translate from test/questions.tex, on 2024-07-09 15:37:18 using model gpt-3.5-turbo-16k
%%

% GPT CHUNK%
\documentclass{ximera}
../preamble.tex
\addPrintStyle{..}

\hintstrue

\begin{document}
    \xmtitle{Questions}{Simple example of exercise and question}

    \begin{exercise} Solve the following questions correctly:
        \begin{question}
                $1+1 = \answer{2}$
        \end{question}    
        \begin{question}
            $1+2 = $ \wordChoice{   \choice{$2$}
                                    \choice[correct]{$3$}
                                    \choice{$4$}
                                }
            \begin{hint}
                By definition, $2 = 1+1$, and addition is associative.
            \end{hint}
            \begin{hint}
                By definition, $1+(1+1) = (1+1) + 1 = 2 + 1 = ?$.
            \end{hint}
            \begin{feedback}[correct] 
                Congratulations, you can already do arithmetic very well! Keep it up. 
            \end{feedback}         
            \begin{oplossing} It holds that
                \[ 1 + 2 = 1 + (1 + 1) = (1 + 1) + 1 = 2 + 1 = 3 \]
                because of the definition $2\perdef 1+1$, the associativity of addition,
                and finally the definition $3\perdef 2+1$. 
            \end{oplossing}
        \end{question}

        \begin{question}$x-4 = 0
            \iff x = \answer{4}$
        \end{question}
        \begin{question}$x-4 = 0
            \iff x = \answer[onlineshowanswerbutton]{4}$
        \end{question}
        \begin{question}$x-4 = 0
            \iff x = \answer[onlinenoinput]{4}$
        \end{question}
        \begin{question}$x^2-4 < 0
            \iff \answer[onlinenoinput]{ x \in \left]-2,2\right[}$
        \end{question}

    \end{exercise}
    \begin{exercise} Tests with answer and feedback

        \begin{question}Write out $3: \answer{drie}$  (using \verb|\answer{drie}|)
        \end{question}

        \begin{question}Write out $3: \answer[format=string]{drie}$ (using \verb|\answer[format=string]{drie}|)
        \end{question}

        \begin{question}Write out $2$: $\answer[id=ansr,format=string]{twee}$ (with tests for feedback)
            \begin{feedback}[correct]{Correct}\end{feedback}
            \begin{feedback}[incorrect]{Incorrect}\end{feedback}    % !!! THIS DOES NOT WORK !!!!
            \begin{feedback}[ansr.toLowerCase() === 'twee']{Correct (because Ximera compares case-insensitive)}\end{feedback}  % but this is equivalent to [correct] ...
            \begin{feedback}[ansr === 'twee']{Correct (according to === in JS)}\end{feedback}  % but this is equivalent to [correct] ...
            \ifonline{
            \begin{feedback}[(ansr.toLowerCase() === 'twee') && (ansr !== 'twee')]{Correct (because Ximera compares case-insensitive) }\end{feedback}   % and this works !!!
            }
            {
            \begin{feedback}[]{Correct (because Ximera compares case-insensitive) }\end{feedback}   % NO && in PDF ...?  
            }
            \begin{feedback}[ansr < 'twee']{Too small}\end{feedback} % this also works, but is probably not very useful 
            \begin{feedback}[ansr > 'twee']{Too big}\end{feedback}
        \end{question}

        \begin{question}
            $\frac{1}{3} =  \answer[tolerance=0.05]{0.33}$  

            Use \verb|\answer[tolerance=0.05]{0.33}| to tolerate rounding errors. You can be off by 0.05 in this case.
        \end{question}

        \begin{question}
            $\frac{1}{3} =  \answer[tolerance=0.05]{\frac{1}{3}}$  

            Use \verb|\answer[tolerance=0.05]{1/3}| to tolerate rounding errors. You can be off by 0.05 in this case.
        \end{question}

        % \begin{question}
        %     $\frac{1}{3} =  \answer[tolerance=0.05]{0.33}$  

        %     Use \verb|\answer[tolerance=0.05]{0.33}| to tolerate rounding errors. You can be off by 0.05 in this case.
        % \end{question}        
        % \begin{question}
        %     $\frac{1}{3} =  \answer[tolerance=0.05]{0,33}$  

        %     Use \verb|\answer[tolerance=0.05]{0.33}| to tolerate rounding errors. You can be off by 0.05 in this case.
        % \end{question}

        \begin{question} Write $x^2: x^{\answer{2}}$
        \end{question}


    \end{exercise}

    \begin{example} Explore which $x\in \R$ satisfy $\left|\dfrac{2x}{3}\right|<1$.

        \begin{basicSkip}
        Solution set: $\answer[onlinenoinput]{\displaystyle
            V=\left]-\frac{3}{2},\frac{3}{2}\right[}$
        \end{basicSkip}
        \begin{oplossing}
            None
        \end{oplossing}
    \end{example}

\end{document}