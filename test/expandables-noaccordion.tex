\documentclass{ximera}
\hintAsExpandablefalse
\expandableAsAccordionfalse
../preamble.tex
\addPrintStyle{..}

\begin{document}
    \xmtitle{Use of expandables (no accordion)}{}

    \begin{example}[First example]
        
        This is an example.

        \begin{hint} With a first hint 
        \end{hint}

        \begin{hint} And a second hint
        \end{hint}

        \begin{feedback} What happens with feedback without questions ... ?
        \end{feedback}
        
        \begin{oplossing} This is the solution of this dummy example
        \end{oplossing}

    \end{example}

    \begin{expandable}{example}{An expandable example}
        
        This is an expandable example.

        \begin{hint} With a first hint 
        \end{hint}

        \begin{hint} And a second hint
        \end{hint}

        \begin{feedback} What happens with feedback without questions ... ?
        \end{feedback}
        
        \begin{oplossing} This is the solution of this dummy example
        \end{oplossing}

    \end{expandable}

    \begin{expandable}{example}{Hint inside hint ...?}
        
        This is test on hint-in-hint 

        \begin{hint} With a first hint 
            \begin{hint} INSIDE of the first hint is this SECOND hint\end{hint}
        \end{hint}

        \begin{hint} And this could then be either a second, or a third hint.
        \end{hint}

        \begin{feedback} What happens with feedback without questions ... ?
        \end{feedback}
        
        \begin{oplossing} This is the solution of this dummy example
        \end{oplossing}

        \begin{example} This a a sub-example
        
            \begin{exercise} with a first exercise: 
                $1+1=\answer{2}$
                \begin{exercise} and a second exercise: $2+2=\answer{4}$

                \end{exercise}
            \end{exercise}
        \end{example}

    \end{expandable}   

\end{document}