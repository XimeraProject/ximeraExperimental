%%
%% Generated by gpt_translate from test/align.tex, on 2024-07-09 16:42:38 using model gpt-3.5-turbo-16k
%%

% GPT CHUNK%
\documentclass{ximera}
../preamble.tex
\addPrintStyle{..}

\begin{document}
    \xmtitle{Use of align}{}

With \verb|$| you get formulas like 
$1\;\textrm{Hz}=\frac{1}{\textrm{s}}$ and $ x =\frac{\frac1x}{1+\frac{1}{x}}$
just within the text,

and with \verb|$$ ... $$| you get formulas in so-called "display mode", centered on a line, like

$$
1\;\textrm{Hz}=\frac{1}{\textrm{s}} \text{ and } x =\frac{\frac1x}{1+\frac{1}{x}}
$$ 

or better with \verb|\[ ..\]|  like
\[
1\;\textrm{Hz}=\frac{1}{\textrm{s}} \text{ and } x =\frac{\frac1x}{1+\frac{1}{x}}
\] 

With \verb|aligned| you can align (here on '='):\quad
$
\begin{aligned}
    1\; \textrm{Hz}   & = \frac{1}{\textrm{s}}          \\
    1 + x             & = \frac{\frac1x}{1+\frac{1}{x}} \\
\end{aligned}
$,

and with \verb|align| you can also do that (in display mode)
\begin{align}
    1\; \textrm{Hz}   & = \frac{1}{\textrm{s}}          \\
    1 + x             & = \frac{\frac1x}{1+\frac{1}{x}} 
\end{align}
and with a final \verb|\\| this becomes
\begin{align}
    1\; \textrm{Hz}   & = \frac{1}{\textrm{s}}          \\
    1 + x             & = \frac{\frac1x}{1+\frac{1}{x}} \\
\end{align}

or with \verb|align*| (no tags)
\begin{align*}
    1\; \textrm{Hz}   & = \frac{1}{\textrm{s}}          \\
    1 + x             & = \frac{\frac1x}{1+\frac{1}{x}} 
\end{align*}

% \begin{onlineOnly}    % DOES not work in PDF (at least in xake2019)
% or perhaps with \verb|align| in mathmode (BUT DOES NOT WORK in PDF!): 
% \[
% \begin{align}
%     1 {\rm Hz}      & = \frac{1}{{\rm s}} \\
%     1 {\rm Hzxxxx}  & = \frac{1}{{\rm xxxxs}} \\
% \end{align}
% \]
% \end{onlineOnly}


You could also use \verb|equation| (MIND THE HTML-PDF difference ?)

\begin{equation}
    1\; \textrm{Hz}   = \frac{1}{\textrm{s}}          \\
    1 + x             = \frac{\frac1x}{1+\frac{1}{x}} \\
    \int_0^\infty \frac{1}{x}  =  \ldots \\
    \int_0^\infty \frac{1}{x^2}  =  \ldots 
\end{equation}
or \verb|equation*|
\begin{equation*}
    1\; \textrm{Hz}   = \frac{1}{\textrm{s}}          \\
    1 + x             = \frac{\frac1x}{1+\frac{1}{x}} \\
    \int_0^\infty \frac{1}{x}  =  \ldots \\
    \int_0^\infty \frac{1}{x^2}  =  \ldots 
\end{equation*}

\begin{example}[The same in an example environment]\nl

    With \verb|$| you get formulas like 
$1\;\textrm{Hz}=\frac{1}{\textrm{s}}$ and $ x =\frac{\frac1x}{1+\frac{1}{x}}$
just within the text,

and with \verb|$$ ... $$| you get formulas in so-called "display mode", centered on a line, like

$$
1\;\textrm{Hz}=\frac{1}{\textrm{s}} \text{ and } x =\frac{\frac1x}{1+\frac{1}{x}}
$$ 

or better with \verb|\[ ..\]|  like
\[
1\;\textrm{Hz}=\frac{1}{\textrm{s}} \text{ and } x =\frac{\frac1x}{1+\frac{1}{x}}
\] 

With \verb|aligned| you can align (here on '='):
$
\begin{aligned}
    1\; \textrm{Hz}   & = \frac{1}{\textrm{s}}          \\
    1 + x             & = \frac{\frac1x}{1+\frac{1}{x}} \\
\end{aligned}
$,

and with \verb|align| you can also do that (in display mode)
\begin{align}
    1\; \textrm{Hz}   & = \frac{1}{\textrm{s}}          \\
    1 + x             & = \frac{\frac1x}{1+\frac{1}{x}} \\
\end{align}

or with \verb|align*| (no tags)
\begin{align*}
    1\; \textrm{Hz}   & = \frac{1}{\textrm{s}}          \\
    1 + x             & = \frac{\frac1x}{1+\frac{1}{x}} \\
\end{align*}

% \begin{onlineOnly}    % DOES not work in PDF (at least in xake2019)
% or perhaps with \verb|align| in mathmode (BUT DOES NOT WORK in PDF!): 
% \[
% \begin{align}
%     1 {\rm Hz}      & = \frac{1}{{\rm s}} \\
%     1 {\rm Hzxxxx}  & = \frac{1}{{\rm xxxxs}} \\
% \end{align}
% \]
% \end{onlineOnly}


You could also use \verb|equation|

\begin{equation}
    1\; \textrm{Hz}   = \frac{1}{\textrm{s}}          \\
    1 + x             = \frac{\frac1x}{1+\frac{1}{x}} \\
    \int_0^\infty \frac{1}{x}  =  \ldots \\
    \int_0^\infty \frac{1}{x^2}  =  \ldots 
\end{equation}
or \verb|equation*|
\begin{equation*}
    1\; \textrm{Hz}   = \frac{1}{\textrm{s}}          \\
    1 + x             = \frac{\frac1x}{1+\frac{1}{x}} \\
    \int_0^\infty \frac{1}{x}  =  \ldots \\
    \int_0^\infty \frac{1}{x^2}  =  \ldots 
\end{equation*}

\end{example}

%TODO: example with tabular, more complex align (e.g., multiple columns)

\end{document}