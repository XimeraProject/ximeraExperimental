%%
%% Generated by gpt_translate from test/align.tex, on 2024-07-09 16:42:38 using model gpt-3.5-turbo-16k
%%

% GPT CHUNK%
\documentclass{ximera}
\newcommand{\dfn}{\textbf}
\renewcommand{\vec}[1]{{\overset{\boldsymbol{\rightharpoonup}}{\mathbf{#1}}}\hspace{0in}}
%% Simple horiz vectors
\renewcommand{\vector}[1]{\left\langle #1\right\rangle}
\newcommand{\arrowvec}[1]{{\overset{\rightharpoonup}{#1}}}
\newcommand{\R}{\mathbb{R}}
\newcommand{\transpose}{\intercal}






\usepackage{mdframed} % For framing content
%\usepackage{ifthen}   % For conditional statements

% Define the 'concept' environment with an optional header
\newenvironment{concept}[1][]{%
  \begin{mdframed}[linecolor=black, linewidth=2pt, innertopmargin=5pt, innerbottommargin=5pt, skipabove=12pt, skipbelow=12pt]%
    \noindent\large\textbf{#1}\normalsize%
}{%
  \end{mdframed}%
}











\colorlet{textColor}{black}
\colorlet{background}{white}
\colorlet{penColor}{blue!50!black} % Color of a curve in a plot
\colorlet{penColor2}{red!50!black}% Color of a curve in a plot
\colorlet{penColor3}{red!50!blue} % Color of a curve in a plot
\colorlet{penColor4}{green!50!black} % Color of a curve in a plot
\colorlet{penColor5}{orange!80!black} % Color of a curve in a plot
\colorlet{penColor6}{yellow!70!black} % Color of a curve in a plot
\colorlet{fill1}{penColor!20} % Color of fill in a plot
\colorlet{fill2}{penColor2!20} % Color of fill in a plot
\colorlet{fillp}{fill1} % Color of positive area
\colorlet{filln}{penColor2!20} % Color of negative area
\colorlet{fill3}{penColor3!20} % Fill
\colorlet{fill4}{penColor4!20} % Fill
\colorlet{fill5}{penColor5!20} % Fill
\colorlet{gridColor}{gray!50} % Color of grid in a plot



\usepackage[utf8]{inputenc}

% 201908/202301: PAS OP: babel en doclicense lijken problemen te veroorzaken in .jax bestand
% (wegens syntax error met toegevoegde \newcommands ...)
\pdfOnly{
    \usepackage[type={CC},modifier={by-nc-sa},version={4.0}]{doclicense}
    \usepackage[dutch]{babel}
}

%
% define softer blue/red/green, use KU Leuven base colors for blue (and dark orange for red ?)
%
% todo: rather redefine blue/red/green ...?
%\definecolor{xmblue}{rgb}{0.01, 0.31, 0.59}
%\definecolor{xmred}{rgb}{0.89, 0.02, 0.17}
\definecolor{xmdarkblue}{rgb}{0.122, 0.671, 0.835}   % KU Leuven Blauw
\definecolor{xmblue}{rgb}{0.114, 0.553, 0.69}        % KU Leuven Blauw
\definecolor{xmgreen}{rgb}{0.13, 0.55, 0.13}         % No KULeuven variant for green found ...

\definecolor{xmaccent}{rgb}{0.867, 0.541, 0.18}      % KU Leuven Accent (orange ...)
\definecolor{kuaccent}{rgb}{0.867, 0.541, 0.18}      % KU Leuven Accent (orange ...)

\colorlet{xmred}{xmaccent!50!black}                  % Darker version of KU Leuven Accent

\providecommand{\blue}[1]{{\color{blue}#1}}    
\providecommand{\red}[1]{{\color{red}#1}}


\providecommand{\isEn}{true}   % set to English

%
\ifdefined\isEn
 \newcommand{\nlen}[2]{#2}
 \newcommand{\nlentext}[2]{\text{#2}}
 \newcommand{\nlentextbf}[2]{\textbf{#2}}
\else
 \newcommand{\nlen}[2]{#1}
 \newcommand{\nlentext}[2]{\text{#1}}
 \newcommand{\nlentextbf}[2]{\textbf{#1}}
\fi

\newcommand\xmsection\subsection
\newcommand\xmsubsection\subsubsection

% Aanpassen printversie
%  (hier gedefinieerd, zodat ze in xourse kunnen worden gezet/overschreven)

\providebool{printbasicversion}
\providebool{printextendedversion}   % not properly implemented
\providebool{printfullversion}       % presumably print everything (debug/auteur)


\providebool{parttoc}
\providebool{printpartfrontpage}
\providebool{printactivitytitle}
\providebool{printactivityqrcode}
\providebool{printactivityurl}
\providebool{printcontinuouspagenumbers}
\providebool{numberactivitiesbysubpart}
\providebool{addtitlenumber}
\providebool{addsectiontitlenumber}
\addtitlenumbertrue
\addsectiontitlenumbertrue


% Commando om de printstyle toe te voegen in ximera's. Zorgt ervoor dat er geen problemen zijn als je de xourses compileert
% hack om onhandige relative paden in TeX te omzeilen
% should work both in xourse and ximera (pre-112022 only in ximera; thus obsoletes adhoc setup in xourses)
% loads global.sty if present (cfr global.css for online settings!)
% use global.sty to overwrite settings in printstyle.sty ...
\newcommand{\addPrintStyle}[1]{
\iftikzexport\else   % only in PDF
  \makeatletter
  \ifx\@onlypreamble\@notprerr\else   % ONLY if in tex-preamble   (and e.g. not when included from xourse)
    \typeout{Loading printstyle}   % logging
    \usepackage{#1/printstyle} % mag enkel geinclude worden als je die apart compileert
    \IfFileExists{#1/global.sty}{
        \typeout{Loading printstyle-folder #1/global.sty}   % logging
        \usepackage{#1/global}
        }{
        \typeout{Info: No extra #1/global.sty}   % logging
    }   % load global.sty if present
    \IfFileExists{global.sty}{
        \typeout{Loading local-folder global.sty (or TEXINPUTPATH..)}   % logging
        \usepackage{global}
    }{
        \typeout{Info: No folder/global.sty}   % logging
    }   % load global.sty if present
    \IfFileExists{\currfilebase.sty}
    {
        \typeout{Loading \currfilebase.sty}
        \input{\currfilebase.sty}
    }{
        \typeout{Info: No local \currfilebase.sty}
    }
    \fi
  \makeatother
\fi
}


\addPrintStyle{..}

\begin{document}
    \xmtitle{Use of align}{}

With \verb|$| you get formulas like 
$1\;\textrm{Hz}=\frac{1}{\textrm{s}}$ and $ x =\frac{\frac1x}{1+\frac{1}{x}}$
just within the text,

and with \verb|$$ ... $$| you get formulas in so-called "display mode", centered on a line, like

$$
1\;\textrm{Hz}=\frac{1}{\textrm{s}} \text{ and } x =\frac{\frac1x}{1+\frac{1}{x}}
$$ 

or better with \verb|\[ ..\]|  like
\[
1\;\textrm{Hz}=\frac{1}{\textrm{s}} \text{ and } x =\frac{\frac1x}{1+\frac{1}{x}}
\] 

With \verb|aligned| you can align (here on '='):\quad
$
\begin{aligned}
    1\; \textrm{Hz}   & = \frac{1}{\textrm{s}}          \\
    1 + x             & = \frac{\frac1x}{1+\frac{1}{x}} \\
\end{aligned}
$,

and with \verb|align| you can also do that (in display mode)
\begin{align}
    1\; \textrm{Hz}   & = \frac{1}{\textrm{s}}          \\
    1 + x             & = \frac{\frac1x}{1+\frac{1}{x}} 
\end{align}
and with a final \verb|\\| this becomes
\begin{align}
    1\; \textrm{Hz}   & = \frac{1}{\textrm{s}}          \\
    1 + x             & = \frac{\frac1x}{1+\frac{1}{x}} \\
\end{align}

or with \verb|align*| (no tags)
\begin{align*}
    1\; \textrm{Hz}   & = \frac{1}{\textrm{s}}          \\
    1 + x             & = \frac{\frac1x}{1+\frac{1}{x}} 
\end{align*}

% \begin{onlineOnly}    % DOES not work in PDF (at least in xake2019)
% or perhaps with \verb|align| in mathmode (BUT DOES NOT WORK in PDF!): 
% \[
% \begin{align}
%     1 {\rm Hz}      & = \frac{1}{{\rm s}} \\
%     1 {\rm Hzxxxx}  & = \frac{1}{{\rm xxxxs}} \\
% \end{align}
% \]
% \end{onlineOnly}


You could also use \verb|equation| (MIND THE HTML-PDF difference ?)

\begin{equation}
    1\; \textrm{Hz}   = \frac{1}{\textrm{s}}          \\
    1 + x             = \frac{\frac1x}{1+\frac{1}{x}} \\
    \int_0^\infty \frac{1}{x}  =  \ldots \\
    \int_0^\infty \frac{1}{x^2}  =  \ldots 
\end{equation}
or \verb|equation*|
\begin{equation*}
    1\; \textrm{Hz}   = \frac{1}{\textrm{s}}          \\
    1 + x             = \frac{\frac1x}{1+\frac{1}{x}} \\
    \int_0^\infty \frac{1}{x}  =  \ldots \\
    \int_0^\infty \frac{1}{x^2}  =  \ldots 
\end{equation*}

\begin{example}[The same in an example environment]\nl

    With \verb|$| you get formulas like 
$1\;\textrm{Hz}=\frac{1}{\textrm{s}}$ and $ x =\frac{\frac1x}{1+\frac{1}{x}}$
just within the text,

and with \verb|$$ ... $$| you get formulas in so-called "display mode", centered on a line, like

$$
1\;\textrm{Hz}=\frac{1}{\textrm{s}} \text{ and } x =\frac{\frac1x}{1+\frac{1}{x}}
$$ 

or better with \verb|\[ ..\]|  like
\[
1\;\textrm{Hz}=\frac{1}{\textrm{s}} \text{ and } x =\frac{\frac1x}{1+\frac{1}{x}}
\] 

With \verb|aligned| you can align (here on '='):
$
\begin{aligned}
    1\; \textrm{Hz}   & = \frac{1}{\textrm{s}}          \\
    1 + x             & = \frac{\frac1x}{1+\frac{1}{x}} \\
\end{aligned}
$,

and with \verb|align| you can also do that (in display mode)
\begin{align}
    1\; \textrm{Hz}   & = \frac{1}{\textrm{s}}          \\
    1 + x             & = \frac{\frac1x}{1+\frac{1}{x}} \\
\end{align}

or with \verb|align*| (no tags)
\begin{align*}
    1\; \textrm{Hz}   & = \frac{1}{\textrm{s}}          \\
    1 + x             & = \frac{\frac1x}{1+\frac{1}{x}} \\
\end{align*}

% \begin{onlineOnly}    % DOES not work in PDF (at least in xake2019)
% or perhaps with \verb|align| in mathmode (BUT DOES NOT WORK in PDF!): 
% \[
% \begin{align}
%     1 {\rm Hz}      & = \frac{1}{{\rm s}} \\
%     1 {\rm Hzxxxx}  & = \frac{1}{{\rm xxxxs}} \\
% \end{align}
% \]
% \end{onlineOnly}


You could also use \verb|equation|

\begin{equation}
    1\; \textrm{Hz}   = \frac{1}{\textrm{s}}          \\
    1 + x             = \frac{\frac1x}{1+\frac{1}{x}} \\
    \int_0^\infty \frac{1}{x}  =  \ldots \\
    \int_0^\infty \frac{1}{x^2}  =  \ldots 
\end{equation}
or \verb|equation*|
\begin{equation*}
    1\; \textrm{Hz}   = \frac{1}{\textrm{s}}          \\
    1 + x             = \frac{\frac1x}{1+\frac{1}{x}} \\
    \int_0^\infty \frac{1}{x}  =  \ldots \\
    \int_0^\infty \frac{1}{x^2}  =  \ldots 
\end{equation*}

\end{example}

%TODO: example with tabular, more complex align (e.g., multiple columns)

\end{document}