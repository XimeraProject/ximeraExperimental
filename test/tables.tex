%%
%% Generated by gpt_translate from test/tables.tex, on 2024-07-09 16:45:56 using model gpt-3.5-turbo-16k
%%

% GPT CHUNK%
\documentclass{ximera}
../preamble.tex
\addPrintStyle{..}

\begin{document}
    \xmtitle{Tables}{}

\ifx\HCode\undefined \else
\Configure{TITLE}{Title contents}
\fi

\begin{example}[align and tag sometimes behave strangely?]
    The following \verb|align| does not resize correctly in MathJAX: it jumps out of the frame for relatively small fonts (resolution).

    The problem likely stems from the \verb|tag|. Moreover, it does not seem to work well with \verb|hyperref|, as a reference with \verb|\hyperref[fail:test_tag]{my text}| online(!) seems to display the text associated with the \verb|label| instead of the text in the \verb|\hyperref|: see \hyperref[fail:test_tag]{this description}.

    Conclusion: avoid \verb|\tag| (or investigate and modify this text accordingly!).

    \begin{align*}
        \important{\frac{a}{b} = \frac{c}{d}}  &\iff \important{ad = bc} 
                     \tag*{equality of (numeric) fractions (cross-multiplication)}\label{fail:test_tag}\\ \\
        \frac{a}{b}+\frac{c}{d} \quad&\perdef\quad \frac{ad+cb}{bd} 
                     \tag*{addition (common denominator)}\label{fail:optelling breuken} \\ \\
        \frac{a}{b} \cdot \frac{c}{d} \quad&\perdef\quad \frac{a\cdot c}{b\cdot d} 
                     \tag*{multiplication (numerator $\times$ numerator, denominator $\times$ denominator) }\label{fail: vermenigvuldiging breuken} \\ \\
        \frac{a}{b} : \frac{c}{d} \quad&\perdef\quad \frac ab \cdot \frac dc = \frac{a\cdot d}{b\cdot c} 
                  \tag*{division (multiply by the reciprocal)}\label{fail: deling breuken}  \\
    \end{align*}
\end{example}

\begin{example}[tabular gets very wide online]
    A \verb|tabular| tends to take up the entire width online. This is usually not desirable. An alternative is to use \verb|array|s, which do resize correctly (though they are in math mode!).

    With \verb|tabular| (and \verb|center|):
    \begin{center}
        \begin{tabular}{lr}
            x & 1 \\
            y & 2 \\
            z & 1
        \end{tabular}
    \end{center}

    With \verb|tabular and @{} | (and \verb|center|):
    \begin{center}
        \begin{tabular}{@{}l@{ }r@{.}}
            x & 1 \\
            y & 2 \\
            z & 1
        \end{tabular}
    \end{center}

    With \verb|tabular and p{} | (and \verb|center|):
    \begin{center}
        \begin{tabular}{|p{1cm}|p{2cm}|p{3cm}|}
            x & 1 & 2\\
            y & 2 & 3\\
            z & 1 & 4
        \end{tabular}
    \end{center}

    With \verb|array|:
    $$
    \begin{array}{l|r}
        x & 1 \\
        \hline
        y & 2 \\
        z & 1
    \end{array}
    $$

    With \verb|array|:
    $$
    \begin{array}{@{XX}l@{|}p{2cm}|}
        x & 1 \\
        \hline
        y & 2 \\
        z & 1
    \end{array}
    $$
\end{example}

\begin{example}\nl

    \begin{tabular}{lcl}
        The & \textbf{domain} $\important{\dom f}$ & of a function $f$ is the set of all \textit{\textbf{admissible} inputs} of the function. \\
        The & \textbf{image} $\important{\bld f}$  & of a function $f$  is the set of all \textit{\textbf{actual} outputs} of the function.
    \end{tabular}

    % \begin{xmdiv}{xmminimaltable}
    %     \begin{tabular}{@{}l@{ }r@{ }l@{ }l@{}}
    %         The & \textbf{domain} & $\important{\dom f}$ & of a function $f$ is the set of all \textit{\textbf{admissible} inputs} of the function. \\
    %         The & \textbf{image} & $\important{\bld f}$  & of a function $f$  is the set of all \textit{\textbf{actual} outputs} of the function.
    %     \end{tabular}
    % \end{xmdiv}

\end{example}
\end{document}