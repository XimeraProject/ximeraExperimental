%%
%% Generated by gpt_translate from test/tables.tex, on 2024-07-09 16:45:56 using model gpt-3.5-turbo-16k
%%

% GPT CHUNK%
\documentclass{ximera}
\newcommand{\dfn}{\textbf}
\renewcommand{\vec}[1]{{\overset{\boldsymbol{\rightharpoonup}}{\mathbf{#1}}}\hspace{0in}}
%% Simple horiz vectors
\renewcommand{\vector}[1]{\left\langle #1\right\rangle}
\newcommand{\arrowvec}[1]{{\overset{\rightharpoonup}{#1}}}
\newcommand{\R}{\mathbb{R}}
\newcommand{\transpose}{\intercal}






\usepackage{mdframed} % For framing content
%\usepackage{ifthen}   % For conditional statements

% Define the 'concept' environment with an optional header
\newenvironment{concept}[1][]{%
  \begin{mdframed}[linecolor=black, linewidth=2pt, innertopmargin=5pt, innerbottommargin=5pt, skipabove=12pt, skipbelow=12pt]%
    \noindent\large\textbf{#1}\normalsize%
}{%
  \end{mdframed}%
}











\colorlet{textColor}{black}
\colorlet{background}{white}
\colorlet{penColor}{blue!50!black} % Color of a curve in a plot
\colorlet{penColor2}{red!50!black}% Color of a curve in a plot
\colorlet{penColor3}{red!50!blue} % Color of a curve in a plot
\colorlet{penColor4}{green!50!black} % Color of a curve in a plot
\colorlet{penColor5}{orange!80!black} % Color of a curve in a plot
\colorlet{penColor6}{yellow!70!black} % Color of a curve in a plot
\colorlet{fill1}{penColor!20} % Color of fill in a plot
\colorlet{fill2}{penColor2!20} % Color of fill in a plot
\colorlet{fillp}{fill1} % Color of positive area
\colorlet{filln}{penColor2!20} % Color of negative area
\colorlet{fill3}{penColor3!20} % Fill
\colorlet{fill4}{penColor4!20} % Fill
\colorlet{fill5}{penColor5!20} % Fill
\colorlet{gridColor}{gray!50} % Color of grid in a plot



\usepackage[utf8]{inputenc}

% 201908/202301: PAS OP: babel en doclicense lijken problemen te veroorzaken in .jax bestand
% (wegens syntax error met toegevoegde \newcommands ...)
\pdfOnly{
    \usepackage[type={CC},modifier={by-nc-sa},version={4.0}]{doclicense}
    \usepackage[dutch]{babel}
}

%
% define softer blue/red/green, use KU Leuven base colors for blue (and dark orange for red ?)
%
% todo: rather redefine blue/red/green ...?
%\definecolor{xmblue}{rgb}{0.01, 0.31, 0.59}
%\definecolor{xmred}{rgb}{0.89, 0.02, 0.17}
\definecolor{xmdarkblue}{rgb}{0.122, 0.671, 0.835}   % KU Leuven Blauw
\definecolor{xmblue}{rgb}{0.114, 0.553, 0.69}        % KU Leuven Blauw
\definecolor{xmgreen}{rgb}{0.13, 0.55, 0.13}         % No KULeuven variant for green found ...

\definecolor{xmaccent}{rgb}{0.867, 0.541, 0.18}      % KU Leuven Accent (orange ...)
\definecolor{kuaccent}{rgb}{0.867, 0.541, 0.18}      % KU Leuven Accent (orange ...)

\colorlet{xmred}{xmaccent!50!black}                  % Darker version of KU Leuven Accent

\providecommand{\blue}[1]{{\color{blue}#1}}    
\providecommand{\red}[1]{{\color{red}#1}}


\providecommand{\isEn}{true}   % set to English

%
\ifdefined\isEn
 \newcommand{\nlen}[2]{#2}
 \newcommand{\nlentext}[2]{\text{#2}}
 \newcommand{\nlentextbf}[2]{\textbf{#2}}
\else
 \newcommand{\nlen}[2]{#1}
 \newcommand{\nlentext}[2]{\text{#1}}
 \newcommand{\nlentextbf}[2]{\textbf{#1}}
\fi

\newcommand\xmsection\subsection
\newcommand\xmsubsection\subsubsection

% Aanpassen printversie
%  (hier gedefinieerd, zodat ze in xourse kunnen worden gezet/overschreven)

\providebool{printbasicversion}
\providebool{printextendedversion}   % not properly implemented
\providebool{printfullversion}       % presumably print everything (debug/auteur)


\providebool{parttoc}
\providebool{printpartfrontpage}
\providebool{printactivitytitle}
\providebool{printactivityqrcode}
\providebool{printactivityurl}
\providebool{printcontinuouspagenumbers}
\providebool{numberactivitiesbysubpart}
\providebool{addtitlenumber}
\providebool{addsectiontitlenumber}
\addtitlenumbertrue
\addsectiontitlenumbertrue


% Commando om de printstyle toe te voegen in ximera's. Zorgt ervoor dat er geen problemen zijn als je de xourses compileert
% hack om onhandige relative paden in TeX te omzeilen
% should work both in xourse and ximera (pre-112022 only in ximera; thus obsoletes adhoc setup in xourses)
% loads global.sty if present (cfr global.css for online settings!)
% use global.sty to overwrite settings in printstyle.sty ...
\newcommand{\addPrintStyle}[1]{
\iftikzexport\else   % only in PDF
  \makeatletter
  \ifx\@onlypreamble\@notprerr\else   % ONLY if in tex-preamble   (and e.g. not when included from xourse)
    \typeout{Loading printstyle}   % logging
    \usepackage{#1/printstyle} % mag enkel geinclude worden als je die apart compileert
    \IfFileExists{#1/global.sty}{
        \typeout{Loading printstyle-folder #1/global.sty}   % logging
        \usepackage{#1/global}
        }{
        \typeout{Info: No extra #1/global.sty}   % logging
    }   % load global.sty if present
    \IfFileExists{global.sty}{
        \typeout{Loading local-folder global.sty (or TEXINPUTPATH..)}   % logging
        \usepackage{global}
    }{
        \typeout{Info: No folder/global.sty}   % logging
    }   % load global.sty if present
    \IfFileExists{\currfilebase.sty}
    {
        \typeout{Loading \currfilebase.sty}
        \input{\currfilebase.sty}
    }{
        \typeout{Info: No local \currfilebase.sty}
    }
    \fi
  \makeatother
\fi
}


\addPrintStyle{..}

\begin{document}
    \xmtitle{Tables}{}

\ifx\HCode\undefined \else
\Configure{TITLE}{Title contents}
\fi

\begin{example}[align and tag sometimes behave strangely?]
    The following \verb|align| does not resize correctly in MathJAX: it jumps out of the frame for relatively small fonts (resolution).

    The problem likely stems from the \verb|tag|. Moreover, it does not seem to work well with \verb|hyperref|, as a reference with \verb|\hyperref[fail:test_tag]{my text}| online(!) seems to display the text associated with the \verb|label| instead of the text in the \verb|\hyperref|: see \hyperref[fail:test_tag]{this description}.

    Conclusion: avoid \verb|\tag| (or investigate and modify this text accordingly!).

    \begin{align*}
        \important{\frac{a}{b} = \frac{c}{d}}  &\iff \important{ad = bc} 
                     \tag*{equality of (numeric) fractions (cross-multiplication)}\label{fail:test_tag}\\ \\
        \frac{a}{b}+\frac{c}{d} \quad&\perdef\quad \frac{ad+cb}{bd} 
                     \tag*{addition (common denominator)}\label{fail:optelling breuken} \\ \\
        \frac{a}{b} \cdot \frac{c}{d} \quad&\perdef\quad \frac{a\cdot c}{b\cdot d} 
                     \tag*{multiplication (numerator $\times$ numerator, denominator $\times$ denominator) }\label{fail: vermenigvuldiging breuken} \\ \\
        \frac{a}{b} : \frac{c}{d} \quad&\perdef\quad \frac ab \cdot \frac dc = \frac{a\cdot d}{b\cdot c} 
                  \tag*{division (multiply by the reciprocal)}\label{fail: deling breuken}  \\
    \end{align*}
\end{example}

\begin{example}[tabular gets very wide online]
    A \verb|tabular| tends to take up the entire width online. This is usually not desirable. An alternative is to use \verb|array|s, which do resize correctly (though they are in math mode!).

    With \verb|tabular| (and \verb|center|):
    \begin{center}
        \begin{tabular}{lr}
            x & 1 \\
            y & 2 \\
            z & 1
        \end{tabular}
    \end{center}

    With \verb|tabular and @{} | (and \verb|center|):
    \begin{center}
        \begin{tabular}{@{}l@{ }r@{.}}
            x & 1 \\
            y & 2 \\
            z & 1
        \end{tabular}
    \end{center}

    With \verb|tabular and p{} | (and \verb|center|):
    \begin{center}
        \begin{tabular}{|p{1cm}|p{2cm}|p{3cm}|}
            x & 1 & 2\\
            y & 2 & 3\\
            z & 1 & 4
        \end{tabular}
    \end{center}

    With \verb|array|:
    $$
    \begin{array}{l|r}
        x & 1 \\
        \hline
        y & 2 \\
        z & 1
    \end{array}
    $$

    With \verb|array|:
    $$
    \begin{array}{@{XX}l@{|}p{2cm}|}
        x & 1 \\
        \hline
        y & 2 \\
        z & 1
    \end{array}
    $$
\end{example}

\begin{example}\nl

    \begin{tabular}{lcl}
        The & \textbf{domain} $\important{\dom f}$ & of a function $f$ is the set of all \textit{\textbf{admissible} inputs} of the function. \\
        The & \textbf{image} $\important{\bld f}$  & of a function $f$  is the set of all \textit{\textbf{actual} outputs} of the function.
    \end{tabular}

    % \begin{xmdiv}{xmminimaltable}
    %     \begin{tabular}{@{}l@{ }r@{ }l@{ }l@{}}
    %         The & \textbf{domain} & $\important{\dom f}$ & of a function $f$ is the set of all \textit{\textbf{admissible} inputs} of the function. \\
    %         The & \textbf{image} & $\important{\bld f}$  & of a function $f$  is the set of all \textit{\textbf{actual} outputs} of the function.
    %     \end{tabular}
    % \end{xmdiv}

\end{example}
\end{document}